\documentclass[12pt, a4paper]{article}
\usepackage[utf8]{inputenc}
\usepackage{fancyhdr}
\usepackage{graphicx}
\usepackage[parfill]{parskip}
\PassOptionsToPackage{hyphens}{url}\usepackage{hyperref}
\usepackage{geometry}
\usepackage[
backend=biber,
style=ieee,
]{biblatex}
\usepackage{kotex}


\pagestyle{fancy}
\fancyhf{}
\setlength{\headheight}{14pt}
\rhead{220818 콜로키움}
\lhead{박석훈}
\cfoot{\thepage}

\begin{document}

\title{생각하고 보고 생각}
\author{Sukhoon Park}
\maketitle

\begin{abstract}

2022. 08. 06. \\
발단: 이거 해야해. 이거 해야해? 이거 굳이 해야해? \\
전개: 이건 뭘 의미해? 이런거 해야해? \\
이 개념은 무엇을 의미해? \\
반대로 이 개념은 뭘 의미하고 있지 않는거야? \\
이 개념은 무엇과 유사해? 그리고 유사한 무엇과는 어떤 부분에서 달라? \\
이 개념은 반드시 존재해야해? \\
이 개념이 잘못 이해하거나 되면 어떤 부작용이 생겨? \\
위기: 이거 할 수 있는거 맞아? \\
절정: 몰라 하래서 한거자나 \\
종말: 어어어어 \\

2022. 08. 07.

내적 행복함수를 찾기. 사상을 대가로 돈을 받았을 때 연명하게 될 학자들의 모습. \\


\end{abstract}

\newpage
\section{개관} \\

조지 허버트 미드의 정신의 사회적 구성(social construction of mind)\footnote{사회적 구성주의가 꼭 사회적 가치를 부정하는 것은 아니다. 니클라스 루만의 경우 핵심적인 사회적 가치, 권리, 규범 등을 인정하지 않고 급진적인 사회적 구성주의를 채택하여, 사회적 현실은 자기생산적으로 소통에 의해 형성되었다고 주장한다. 사회이론에서는 ontology가 사회과학을 social constructionism 혹은 constructivism으로 바라본다고 일컫는다.}\footnote{social constructionism과 social constructivism의 차이에 대해서는 후술} 태제라 함은 정신(mind)을 인간 개인의 속성이 아니라, 인간들 사이의 소통과 사회적 경험의 과정에서 등장하여 발전되는 현상으로 보는 것(하홍규, 2011)을 의미한다. 켄 플루머(Ken Plummer)는 상징적 상호작용론은 크게 네 개의 테마로 이루어져있다고 서술한다. 첫째, 인간세계는 분명히 물질적인 객관적 세계일뿐만 아니라 거대한 상징세계이기도 하다\footnote{이는 상징적 상호작용론이 생리적 혹은 물질적인 존재를 간과한다는 일반적인 오해에서 벗어나는 핵심적인 주장이기도 하다}. 상호작용론자들이 볼 때, 인간과 여타 동물이 다른 것은 인간이 정교한 기호학 - 즉 인간들에게 역사, 문화 그리고 매우 복잡하고 모호한 의사소통만을 산출할 수 있게 해 주는 상징생산능력 - 을 가지고 있기 때문이다\footnote{조지 허버트 미드의 책을 중심으로 일부 구절 인용 예정}. 두 번째는 삶과 상황 그리고 심지어는 사회도 항상 그리고 어디서나 지속적으로 전개되고 적응되고 형성되는 와중에 있다. 이러한 지속적인 과정이 상호작용론자들로 하여금 자아감의 획득, 일생의 과정, 타자에 대한 적응, 시간감각의 조직, 질서협상, 문명의 구축 등의 전략에 초점을 맞추게 한다. 셋째로, 상호작용론자들의 모든 저작은 개인이나 사회 그 자체에 초점을 맞추지 않는다. 오히려 그것의 관심은 삶을 조직하고 사회를 구성하는 연합행동(joint acts)이다. 그것은 `집합행동(collective behavior)\footnote{사회학에서 처음으로 collective라는 단어를 사용한 사람은 에밀 뒤르켐(Emile Durkheim)으로 알려져있다. 뒤르켐은 집합의식(collective consciousness) 개념에서 사회현상 가운데에는 개인의식으로 환원하여 파악할 수 없는, 개인의식 외부에 존재하는 객관적 징표에 의한 것들이 있음을 증명하고 있다(뒤르켐:1992. 166). Gustave Le Bon은 ``심리학적 군중" 개념에서 군중 속의 개인은 그들이 고립되어 있을 때 느끼고 생각하고 행동하는 것과는 확연히 구분되는 특징을 갖는다고 설명하며, 그는 군중속에서 개인이 그들의 자유의지와 이성적으로 행동하는 능력을 상실한다고 믿는다. 이러한 Le Bon의 주장은 Sigmund Freud가 느꼈던 개인과 군중의 관계와도 일치한다.}'에 관심을 갖는다. 그것의 가장 기본적 개념이 자아(self)이다. 자아는 `타자(the other)'라는 관념이 항상 삶 속에 존재한다는 것을 함축한다. 우리는 결코 `나' 홀로 존재할 수 없다(Wiley, 1994, 재인용). 따라서 상호작용이론의 모든 핵심 사상과 개념은 항상 개인에 영향을 미치는 이 같은 사회적 타자(social other)를 강조한다. 사실 `개인'이라는 관념 자체는 타자를 통해 구성된다. 상호작용이론은 기본적으로 ``사람들이 서로 영향을 미치는 방식"에 관심을 기울인다(Becker, 1996, 재인용). 네 번째 테마는 경험세계에 참여하는 것과 관련된다. 이론적 창공으로 비상할 수도 있는 다른 많은 사회이론과는 달리, 상징적 상호작용론자들은 지상에 머무른다.

\newpage

\section{정신, 자아 그리고 사회 강독} \\
사회적 현상으로서 정신 이해는 순수하게 개인적이고 합리적인 존재로 상정되는 르네 데카르트의 정신 이해와 대비된다(하홍규, 2011). 미드는 정신이 사회적 산물이며, 인간은 타인과 관계를 맺음으로써 존재하기 때문에 이성적 존재가 된다. 데카르트의 경구처럼 `나는 생각한다, 그러므로 존재한다'가 아니라, 미드에게 있어서는 `함께 존재한다. 그러므로 나는 생각한다'.

(75p) 내가 제안하고 싶은 이론은 경험을 사회의 관점에서, 적어도 사회적 질서에 필수적인 커뮤니케이션의 관점에서 다루는 것이다. 이 관점에서 보면 사회심리학은 경험에 접근하는 것을 개인의 관점에서 미리 가정하지만, 개인 자신이 사회적 구조, 사회적 질서 안에 속해 있기 때문에 특히 어떤 것들이 이 경험에 속하는지를 결정해야 한다.

(75-76) 정신과 자아는 본질적으로 사회적 산물인 반면, 인간 경험의 사회적 측면의 산물과 현상, 즉 경험의 바탕이 되는 생리적 기제는 유전 및 존재와 결코 무관할 수 없고 사실 필수적이다. 왜냐하면 물론 개인의 경험과 행동은 사회적 경험과 행동에 생리적 기초가 되지만, 사회적 경험과 행동(정신과 자아의 근원 및 존재에 필수적인 것도 포함됨)도 개인적 경험과 행동 그리고 이것들의 사회적 기능에 생리적으로 의존하기 때문이다.

(79-80) 두 유형의 행동주의에서 모두 사물이 가지고 있는 특성과 개인이 가지고 있는 특성이 하나의 행위 안에서 발생하는 것이라고 할 수 있다. 그러나 그 행위의 일부는 유기체 안에 있고, 나중에 가서야 표현된다. 내가 볼 때 이 부분이 바로 왓슨\footnote{https://www.goodtherapy.org/famous-psychologists/john-watson.html}이 간과했다고 생각되는 행동의 측면이다. 행위 자체 안에 외적이지는 않지만 그 행위에 속하는 영역이 있고, 우리 자신의 태도로 스스로를 나타내는 내적 유기체 행위의 특성, 특히 말과 관련되는 특성들이 있다.

(80) 우리는 내적 의미가 표현되는 관점에서가 아니라, 신호와 제스처의 의미를 통해 더 넓은 범위에서 일어나는 집단 속의 협동 맥락에서 언어를 분석하고자 한다. 의미는 이러한 과정 안에서 나타난다. 우리의 행동주의는 사회적 행동주의다.

(84-85) 왓슨은 객관적으로 관찰가능한 행동이 완전하게 그리고 배타적으로 과학적 심리학의 영역을 구성한다고 주장한다. 그는 '정신'(mind) 또는 '의식'(consciousness)은 잘못된 것으로 제쳐놓고, 모든 '정신적' 현상을 조건반사와 이와 유사한 생리적 기제, 즉 순수하게 행동주의적인 측면으로 환원하고자 한다. 물론 이러한 시도는 잘못된 것이고 성공적이지 못하다. 왜냐하면 정신이나 의식 같은 존재는 어떤 의미에서 수용되어야 하기 때문이다.

(88-89) 언어는 사회적 행동의 일부다. '언어'의 역할을 할 수 있는 부호화 상징의 수는 무한히 많다. 우리는 다른 사람들의 행동의 의미를 읽는데, 아마도 이것을 인식하지는 못하는 것 같다. 

(89) 사실, 앞으로 보게 되겠지만, 언어는 그와 똑같은 과정으로 발생한다. 그러나 우리는 언어에 접근할 때 마치 철학자들이 하듯이 사용되는 상징의 관점에서 접근하는 경향이 농후하다. 우리는 그 상징을 분석하여 그 상징을 사용하는 개인의 마음속에 무슨 의도가 있는지를 알아낸 다음, 이 상징이 상대방의 마음속에 있는 것 같은 의도를 불러일으키는지를 찾아내고자 한다.

(92-93) 그러나 다윈과 반대로, 우리는 한 유기체 쪽에서 행동을 일으킬 때 다른 유기체 쪽에서 그 행동에 의존하지 않고 적응 반응을 불러일으킬 수 있는 그런 종류의 의식이 사전에 존재한다는 증거를 찾을 수 없다. 우리는 오히려 어쩔 수 없이 의식은 그러한 행동에서 출현한다고 결론지을 수밖에 없다. 의식이 사회적 행위의 전제조건이 아니라, 사회적 행위가 의식의 전제조건이다. 그 행위 안에 분리된 요소로서 의식의 개념을 끌어들이지 않고도 사회적 행위의 메커니즘을 추적할 수 있다. 따라서 사회적 행위는 의식의 형태 없이 또는 그와 별개로, 더 기초적 단계 또는 형태로 존재할 수 있다.

(105-106) 나는 `의식'이라는 용어를 어떤 내용에 대한 접근 가능성을 나타내기 위해 사용하는 것과 그 내용 자체와의 동의어로 사용하는 것을 구분하고 싶다. 눈을 감으면 당신은 어떤 자극들로부터 스스로 차단하는 것이다. 마취제를 취하면 세상은 당신에게 접근가능하지 않다. 이와 유사하게 수면도 사람을 세상에 접근불가능하게 만든다. 나는 이처럼 어떤 영역에 사람이 접근가능하거나 접근불가능하게 만드는 것을 지칭하는 뜻으로 의식이라는 말을 사용하는 경우와 개인의 경험에 의해 결정되는..내용 자체를 뜻하는 것으로 사용하는 경우를 구분하고자 한다.

(109) 어떤 것이 그 개인에게만 접근가능한지, 그 자신의 내적 생활 영역에만 일어나는 것이 무엇인지 하는 것은 그것이 발생하는 상황과의 관련성 속에서 설명되어야 한다. 한 개인은 한 경험을, 또 다른 개인은 또 다른 경험을 가지고 있고, 이 둘은 모두 각자의 생활내력에 의해 설명된다. 그러나 그밖에도 모든 사람의 경험에 공통되는 것이 있다. 그리고 우리의 과학적 진술은 개인 자신이 경험한 것, 그래서 궁극적으로 자신의 경험만으로 설명될 수 있는 것과 모든 사람에게 속한 경험 사이의 상관관계를 설명한다.

(120) 나는 제스처를 설명하는 방법으로 개들이 싸우는 장면의 예를 들어왔다. 한 마리 개의 제스처는 다른 개의 반응을 나오게 하는 자극 역할을 한다. 그러면 두 마리 개 사이에 관계가 형성된다. 그리고 상대 개의 행위에 대해 처음 개가 또 반응을 하기 때문에, 두 마리 개는 교대로 변화를 겪는다. 한 마리 개가 다른 개를 공격할 준비가 되어 있다는 사실 자체가 다른 개로 하여금 자기 위치나 태도를 변화시키는 원인이 된다. 우리는 여기서 제스처의 대화를 볼 수 있다. 그러나 이것은 의미 있는 제스처는 아니다.

(123) 이제 제스처가 그 이면의 생각을 의미할 때, 그리고 그것이 다른 개인의 마음속에 있는 생각을 불러일으킬 때, 우리는 의미 있는 상징을 갖게 된다. 개가 싸우는 경우에서 우리는 적절한 반응을 불러내는 제스처를 보았다. 그러나 이번에는 우리가 첫 번째 개인의 경험 안에 있는 의미에 대응하고, 또한 두 번째 개인 안에 있는 의미를 불러내는 그런 상징을 갖는다. 제스처가 거기에까지 미치는 경우, 이것을 우리는 '언어'라고 하게 된다. 여기서 제스처는 `의미 있는 상징'으로서 어떤 의미를 부여받게 되는 것이다.

(124) 그리고 의미 있는 제스처 또는 의미 있는 상징은 의미를 부여받지 못한 제스처보다 그러한 적응에 훨씬 더 큰 혜택을 주는데, 그 이유는 의미 있는 제스처나 상징이 그것을 내보내는 개인이나 그와 함께 주어진 사회적 행위 안에서 참가하는 다른 개인들에게서 그 제스처(또는 그 의미)에 대한 똑같은 태도를 끌어내고, 이로써 그 제스처에 대한 태도를 그가 (자기 행동의 구성요소로서) 의식할 수 있도록 만들어서 그 태도에 비추어 그들에게 나중의 행동을 적응할 수 있게 하기 때문이다. 간단히 말해서, 의식적 또는 의미 있는 제스처의 대화는 비의식적 또는 의미 없는 제스처의 대화보다 사회적 행위-그 제스처를 하는 개인들 각자가 자기에 대한 상대방의 태도를 알아차리는 것-안에서 상호 적응하는데 훨씬 더 적절하고 효과적인 메커니즘이다.

(125) 단지 의미 있는 상징으로서의 제스처 측면에서만 마음이나 지능의 존재가 가능하다. 단지 의미 있는 상징으로서의 제스처일 때에만 생각 - 이런 제스처를 통해 개인이 자기 자신과 내면화된 또는 암묵적인 대화를 하는 것이라고 간단히 정의됨 -이 일어난다. 우리가 사회적 과정에서 다른 개인들과 함께 진행하는 제스처의 외적인 대화를 경험함으로써 내면화되는 것이 생각의 핵심이다. 제스처는 이렇게 내면화되어서 의미 있는 상징이 되는데, 그 이유는 주어진 사회 또는 사회적 집단의 모든 개개 구성원들에게 같은 의미를 지니게 되기 때문이다. 즉 의미 있는 상징이 된 제스처는 그 제스처를 하는 사람과 거기에 반응하는 사람에게 똑같은 태도를 일으키기 때문이다. 그렇지 않으면 개인은 그것을 내면화할 수도 없고 그것을 의식하여 의미를 파악할 수도 없을 것이다.

(137-138) 유기체의 입장에서 단순히 다른 유기체의 행동을 보거나 듣는 것을.. 재생산해내는 경향으로서의 모방은 기계적으로 불가능하다. 유기체에게 도달하는 모든 시각적, 청각적 자극이 경험에서 보고 들은 것을 재생산해내게 할 정도로 잘 정비되어 있는 유기체는 상상할 수 없다.

(144) 다른 사람이 느끼게 하기 위해서 보이는 수단으로 신체 표현을 사용하는 사람은 배우뿐이다. 그는 거울을 끊임없이 사용함으로써 자기가 어떻게 보일지를 스스로에게 보여주는 반응을 취한다. 그는 분노, 사랑, 이것, 저것 또는 다른 태도를 새기며, 거울을 통해 스스로 자기가 어떻게 그렇게 보일 수 있는지를 알아본다. 나중에 그가 제스처를 사용할 때 이것은 정신적 이미지로서 존재한다. 그는 그 특별한 표현이 공포를 불러일으킨다는 사실을 인식한다. 만약 우리가 음성 제스처를 배제한다면, 그것은 단지 다른 사람이 반응하는 것처럼 스스로 제스처에 반응하는 위치에 도달하려고 거울을 사용할 때만 그렇다. 그러나 음성 제스처는 다른 사람이 대응하는 것처럼 스스로의 자극에 대응하는 능력을 부여하는 것이다. 

(148) 우리는 특히 음성 제스처를 사용해서 우리가 다른 사람들에게서 불러내는 반응을 계속해서 스스로에게서도 불러내며, 다른 사람들이 우리 행동에 대해 지니는 태도를 우리 스스로도 취하게 된다.

(148) 왓슨과 같은 행동주의자는 우리의 모든 사고과정이 음성화되었다고 주장한다. 생각할 때 우리는 단순히 어떤 단어를 사용하기 시작한다. 그것은 어떤 의미에서 사실이다. 그러나 왓슨은 여기에 포함되는 모든 것을 고려하지 않았다. 즉 이 자극이 정교한 사회적 과정에서 본질적인 요소가 되어 사회적 과정에 가치를 부여한다는 사실을 간과한 것이다. 그러한 음성적 과정은 이처럼 아주 중요하며, 음성적 과정은 그것과 어울리는 지능, 사고과정과 함께 단순히 서로에 대해 특별한 음성적 요소를 실행하는 것만이 아니다. 이런 관점은 언어의 사회적 배경을 무시한 것이다.

(152) 우리는 동물들이 생각하지 않는다고 말한다. 동물은 자기가 책임져야 할 위치에 자기 자신을 놓지 않는다. 동물은 다른 쪽 상대방의 입장에 자기를 놓고 그 결과로서 ``그는 그런 식으로 행동할 것이고, 그러면 나는 이런 식으로 행동할 것이다"라고 생각하지 않는다. 

(158) 이와 유사하게, 사회적 과정은 어떤 의미에서 그것이 반응하는, 또는 적응하는 대상들을 구성한다. 다시 말해서, 대상들은 사회적 과정에서 서로에게 적응하며 반응하거나 행동하는 다양한 개인 유기체들을 통해 경험과 행동의 사회적 과정의 이른 진화 단계에서 제스처의 대화 형태, 그리고 진화의 나중 단계에서는 언어의 형태를 취하는 커뮤니케이션 수단을 통해 가능해진다. 

(162) 의미의 의미 문제에 관해 많이 민감하게 다루어왔다. 이 문제를 해결하기 위한 시도에서 정신적 상태에 의지할 필요는 없다. 우리가 보아왔듯이, 의미의 본질은 사회적 행위의 구조 안에 내재되어 있다는 사실이 밝혀져왔기 때문이다. 구체적으로, 그 의미는 3종의 기본적인 개개 요소들 사이의 관계 안에 내재되어 있는데, 즉 한 개체의 제스처, 그 제스처에 대한 두 번째 개체의 반응, 그리고 첫 번째 개체의 제스처에 의해 시작된 해당 사회적 행위의 완성이 3종의 기본 요소다. 따라서 의미의 본질이 사회적 행위의 구조 속에 내재되어 있다는 사실 때문에 사회심리학 안에서 필연성을 추가로 강조해야 할 필요가 있는데, 그것은 바로 어떤 개인들의 집단이 포함되어 진행 중인 경험과 행동의 사회적 과정을 처음부터 가정하고 시작해야 한다는 것이다. 그 바탕 위에서 개인들의 정신, 자아 그리고 자기의식의 존재와 발달이 가능해진다.

(205-206) 우리는 한 세트의 상징을 가지고 있어서, 그것을 통해 어떤 특성들을 지시하며, 그 특성들을 지시하면서 그것들을 즉각적인 환경에서 분리해내고, 단순히 하나의 관계를 명확히 유지한다. 우리는 곰의 발자국을 분리하여 그 발자국을 만든 동물과의 관계에만 유지한다. 우리는 그것에 반응하는 것이지, 다른 데 반응하는 것이 아니다. 우리는 곰을 지시하는 것으로서 그것에 의미를 부여하며, 피하거나 사냥해야 할 것으로 경험 안에서 그 대상이 지니는 가치를 부여한다. 대상에 대한, 그리고 그 대상에 속하는 반응에 대한 관계 안에서 이와 같은 중요한 특성들을 분리하는 능력은 우리가 일반적으로 인간이 사물에 관해 생각한다고, 또는 정신을 가지고 있다고 말할 때 의미하는 것이다. 그러한 능력은 상징이라는 수단을 통한 인간의 사고과정을 흰 쥐의 경우에서 반사의 조건화..와 아주 큰 차이를 가져오게 한다.

(208-209) 하등동물의 경우에는 그런 상황에 대한 증거가 없다. 어떤 동물/행동에서도 상징, 커뮤니케이션 방법, 다른 반응들에 대응하는 것을 상세히 실현할 만한 것을 찾을 수 없으며, 따라서 개체의 경험 안에 모든 가능성을 지니고 있을 수 없다는 사실에서 이것은 분명해진다. 반사적 지능을 가진 존재를 하등동물의 행위에 구분해주는 것은 바로 이 부분이다. 그리고 그것을 가능하게 해주는 메커니즘은 언어다. 우리는 언어가 행위의 일부임을 인식해야 한다. 그러나 정신에는 사물의 특성에 대한 관계가 포함된다. 그 특성들은 사물 속에 있으며, 반면에 자극은 어떤 의미에서 유기체 속에 존재하는 반응, 밖에 존재하는 사물에 대한 반응을 불러일으킨다. 모든 과정이 다 정신적 산물인 것은 아니며, 따라서 뇌 안에서만 원인을 찾을 수 있는 것도 아니다. 정신성은 바로 상징들의 집합이 매개하는 상황에 유기체가 관련을 맺을 때 발생한다.

\section{미드 원서 중 일부 내용}

Mead. G. H. 1967. Mind, Self & Society. Chicago.

그 스스로 객체가 될 수도 있는 자아는, 본질적으로 사회구조이다. 그 자아는 사회 경험의 과정에서 생성된다. 사회 경험이 전제되지 않은 자아의 생성은 상상조차 할 수 없다(Mead, 1967:140)

적응과 재적응이 반복된다. 우리가 어떠한 사람에 대해 이야기할 때 그는 낯설지 않은 개인이다. 그의 생각은 정확하게 그의 이웃들의 생각과 일치한다. 그는 주어진 상황 하에서 거의 객관적 자아 이상의 것이 아니다. 그의 적응은 거의 무의식적으로 일어나는 적응일 뿐이다(Mead, 1967: 200)

미드에게 사회 경험이란 개인이 공동체의 반응을 고려하고 그에 적응하는 과정이다. 사회 경험은 같은 사회 경험에 참여하고 있는 공동체의 조직된 기대, 일반화된 타자들의 기대에 개인 스스로 적응하는 과정이고, 그 기대를 자신의 목적격 자아(me)에 수용하는 과정이다. 그 기대에 적응함으로써, 그리고 그 기대를 자신의 목적격 자아에 수용함으로써 개인은 자신이 속한 공동체의 일원임을 스스로 확인한다. (53)

이 인용문처럼 일반화된 타자들의 사회 기대·집합적 힘을 뚜렷하게도덕적인 성격으로 규정하고 있는 뒤르켐과는 달리, 조지 허버트 미드는 ``정신, 자아 그리고 사회"에서 일반화된 타자들의 사회 기대·집합적힘의 성격을 분명하게 규정하고 있지 않다. 그는 조직된 타자, 일반화된 타자가 규범을 지향하고 있는지에 대해서, 그리고 공동체의 반작용이 도덕적인 것인지에 대해서 분명하게 언급하고 있지 않다. 곧 조직된 타자, 일반화된 타자가 개인에게 도덕적인 것과 비도덕적인 것가운데 어느 것을 요구하고 있는지 분명하지 않다(61)

\newpage

\section{보론. 뒤르켐의 종교생활의 원초적 형태}
뒤르켐(E. Durkheim). 1992. 종교 생활의 원초적 형태. 노치준, 문혜숙 옮김. 민영사

사회학 이론에서 사회가 개인의식의 산물이 아니라 집합 의식의 산물이라는 사실을 최초로 논증한 사람은 뒤르케임이다. 그는 ``사회학적방법의 규칙들''에서 개인의식과 구분되는 집합 의식의 존재를 사회학적으로 최초로 분류해내었다. 그는 사회현상 가운데에는 개인의식으로 환원하여 파악할 수 없는, 개인의식 외부에 존재하는 객관적 징표에 의한 것들이 있음을 증명하고 있다. 그는 집합 의식이 개인의식의단순한 합이라는 주장이나 사회가 개인의식의 단순한 합이라는 주장과는 정반대로, 집합 의식이 거꾸로 개인의식에 영향을 미친다고 설명한다. 사회현상의 기본적인 특성은 개인의식에 압력을 행사하는 힘이기 때문에 사회현상은 개인의식에서 나오는 것이 아니며, 결과적으로 사회학은개인 심리학의 추론이 아니라는 결론이 나온다. 왜냐하면 이러한 구속의힘은 사회현상이 우리와는 아주 다른 성격을 가지고 있다는 사실의 증거이며, 이는 사회현상이 강제에 의해 혹은 적어도 다소 과중한 무게감에의해 우리를 통제하기 때문이다(뒤르케임, 1992: 166).

이러한 행위의 양식이나 사고의 양식은 개인의 외부에 있을 뿐만 아니라더욱이 강제력을 가지며 그러한 강제력을 통하여 개인의 의지와는 독립적으로 개인에게 그러한 양식을 부과한다. 물론 내가 그러한 양식을 수용하고 따를 때 그 강제력을 아주 가볍게 느끼고 완전히 따른다면, 강제력은 필요 없게 된다. 강제력은 사실의 내부적 속성이지만, 그것에 저항하려는 순간 자신을 강제함으로써 그것의 존재는 증명된다(뒤르케임, 1992: 54).

\newpage
\subsection{정신, 자아 그리고 사회 1부와 2부 220818}

제 1장 사회심리학과 행동주의

(75p) 내가 제안하고 싶은 이론은 경험을 사회의 관점에서, 적어도 사회적 질서에 필수적인 커뮤니케이션의 관점에서 다루는 것이다. 이 관점에서 보면 사회심리학은 경험에 접근하는 것을 개인의 관점에서 미리 가정하지만, 개인 자신이 사회적 구조, 사회적 질서 안에 속해 있기 때문에 특히 어떤 것들이 이 경험에 속하는지를 결정해야 한다. \\

(75-76) 정신과 자아는 본질적으로 사회적 산물인 반면, 인간 경험의 사회적 측면의 산물과 현상, 즉 경험의 바탕이 되는 생리적 기제는 유전 및 존재와 결코 무관할 수 없고 사실 필수적이다. 왜냐하면 물론 개인의 경험과 행동은 사회적 경험과 행동에 생리적 기초가 되지만, 사회적 경험과 행동(정신과 자아의 근원 및 존재에 필수적인 것도 포함됨)도 개인적 경험과 행동 그리고 이것들의 사회적 기능에 생리적으로 의존하기 때문이다. 

(76) 심리학 자체는 의식의 영역만을 연구하는 학문이 될 수 없다. 더 포괄적인 영역의 연구가 필수적이다. 그러나 심리학은 다른 과학이 다루지 않는 현상, 즉 개인 자신만이 경험적으로 접근할 수 있는 현상을 연구하기 위해 개인의 경험 안을 들여다본다는 점에서 내성을 이용하는 과학이다.

(76-77) 따라서 심리학적 자료는 '접근가능성'(accessibility)의 측면에서 가장 잘 정의될 수 있다. 개인의 경험 안에서 단지 그 개인만이 접근할 수 있는 것이 특히 심리학적인 것이다.

cf) 행위와 행동의 차이

(79-80) 두 유형의 행동주의에서 모두 사물이 가지고 있는 특성과 개인이 가지고 있는 특성이 하나의 행위 안에서 발생하는 것이라고 할 수 있다. 그러나 그 행위의 일부는 유기체 안에 있고, 나중에 가서야 표현된다. 내가 볼 때 이 부분이 바로 왓슨이 간과했다고 생각되는 행동의 측면이다. 행위 자체 안에 외적이지는 않지만 그 행위에 속하는 영역이 있고, 우리 자신의 태도로 스스로를 나타내는 내적 유기체 행위의 특성, 특히 말과 관련되는 특성들이 있다.

(80) 우리는 내적 의미가 표현되는 관점에서가 아니라, 신호와 제스처의 의미를 통해 더 넓은 범위에서 일어나는 집단 속의 협동 맥락에서 언어를 분석하고자 한다. 의미는 이러한 과정 안에서 나타난다. 우리의 행동주의는 사회적 행동주의다.

(80) 사회심리학은 사회적 과정 안에 놓여 있는 그대로의 개인의 행위나 행동을 연구한다. 한 개인의 행동은 그가 구성원으로 있는 전체 사회집단의 행동 측면에서만 이해될 수 있는데, 그 이유는 그의 개인적 행위가 자신을 넘어서는 좀더 큰 사회적 행위, 따라서 그 집단의 다른 구성원들을 시사하는 행위 속에 포함되기 때문이다.

(80-81) 사회심리학에서 우리는 사회를 구성하는 분리된 개인들의 행동 측면에서 사회적 집단의 행동을 바라보지 않는다. 오히려 복잡한 사회적 집단 활동 전체에서 출발하여 (구성요소로서) 집단을 구성하는 개별적 개인들 각각의 행동을 분석한다. 즉 우리는 개인의 행위를 사회적 집단의 조직화된 행위로 설명하려고 하며, 사회적 집단의 조직화된 행위를 거기에 속하는 개별적 개인들의 행동으로 설명하려고 하지 않는다. 사회심리학에서 전체(사회)는 부분(개인)에 선행하며, 부분이 전체보다 선행하지 않는다. 그리고 부분은 전체의 측명네서 설명되며, 전체가 부분이나 부분들로 설명되지 않는다. 


제 2장 태도의 행동주의적 의미

(84-85) 왓슨은 객관적으로 관찰가능한 행동이 완전하게 그리고 배타적으로 과학적 심리학의 영역을 구성한다고 주장한다. 그는 '정신'(mind) 또는 '의식'(consciousness)은 잘못된 것으로 제쳐놓고, 모든 '정신적' 현상을 조건반사와 이와 유사한 생리적 기제, 즉 순수하게 행동주의적인 측면으로 환원하고자 한다. 물론 이러한 시도는 잘못된 것이고 성공적이지 못하다. 왜냐하면 정신이나 의식 같은 존재는 어떤 의미에서 수용되어야 하기 때문이다.

(85) 그러나 정신이나 의식을 순수하게 행동주의적인 측면으로 환원하는 것이 불가능하다 할지라도 행동의 측면에서 의식을 설명하는 것은 불가능하지 않다. 최소한 의식의 존재를 부정하거나 설명에서 제외하지 않고 행동 측면에서 설명하는 것은 가능하다... 그러나 그 반대로 우리는 다른 의미에서 의식의 존재를 부정하지 않고 심리적 실체로서의 존재를 부정한다. 그래서 우리가 의식을 기능적으로, 즉 초월적 현상이 아닌 자연적 현상으로 간주한다면, 의식을 행동주의적 용어로 다룰 수 있게 된다. 

(86) 그러나 현재의 결과들을 살펴보면 행위가 태도의 측면에서 조직됨을 알 수 있다. 행위를 담당하는 신경계의 여러 부분이 조직화되어 있는데, 이 조직은 바로 일어나는 것뿐만 아니라 일어나려고 하는 다음 단계까지 나타낸다. 만약 사람이 멀리 떨어져 있는 대상에 접근한다면, 그가 거기에 도착한 다음에 무엇을 하려고 하는지를 염두에 두며 접근하는 것이다. 만약 망치에 접근한다면 그의 근육은 모두 그 망치의 손잡이를 잡을 준비를 하는 것이다. 행위의 나중 단계가 이른 단계 속에 존재하는데, 이것들이 단순히 모두 곧 시작될 준비를 한다는 의미에서만이 아니라 그 과정 자체를 통제하는 역할을 한다는 의미에서 그렇다. 나중 단계들은 우리가 그 대상에 어떻게 접근할 것인지, 그리고 그것을 먼저 조작하는 단계를 결정한다. 그래서 우리는 중추신경계 안의 어던 세포들의 집합이 내면화되어, 이미 행위의 나중 단계를 미리 주도할 수 있다는 사실을 깨달을 수 있다. 여기서 행위는 하나의 전체로서 그 과정을 결정할 수 있다. 

제 3장 제스처의 행동주의적 의미

(88-89) 언어는 사회적 행동의 일부다. '언어'의 역할을 할 수 있는 부호화 상징의 수는 무한히 많다. 우리는 다른 사람들의 행동의 의미를 읽는데, 아마도 이것을 인식하지는 못하는 것 같다. 

(89) 사실, 앞으로 보게 되겠지만, 언어는 그와 똑같은 과정으로 발생한다. 그러나 우리는 언어에 접근할 때 마치 철학자들이 하듯이 사용되는 상징의 관점에서 접근하는 경향이 농후하다. 우리는 그 상징을 분석하여 그 상징을 사용하는 개인의 마음속에 무슨 의도가 있는지를 알아낸 다음, 이 상징이 상대방의 마음속에 있는 것 같은 의도를 불러일으키는지를 찾아내고자 한다.

(89) 주1. 사회적 과정이 진행되는 기본적 기제는 무엇인가? 그것은 제스처의 기제로서, 사회적 과정 안에 포함되어 있는 서로 다른 개인 유기체들의 행동에 대한 적절한 반응들을 가능하게 해준다. 어떤 주어진 사회적 행위 안에서 제스처라는 수단을 통해, 한 유기체의 행위와 다른 유기체의 행위 사이에 이루어지는 적응이 효과적으로 나타난다.  

(89-90) 우리는 사람들의 마음속에 생각의 집합이 있다고 가정하며 이 개인들이 가지고 있는 의도에 응답하는 어던 임의적 상징들을 사용한다고 가정한다. 

Charles Darwin "Expression of the Emotions in Man and Animals"

(92-93) 그러나 다윈과 반대로, 우리는 한 유기체 쪽에서 행동을 일으킬 때 다른 유기체 쪽에서 그 행동에 의존하지 않고 적응 반응을 불러일으킬 수 있는 그런 종류의 의식이 사전에 존재한다는 증거를 찾을 수 없다. 우리는 오히려 어쩔 수 없이 의식은 그러한 행동에서 출현한다고 결론지을 수밖에 없다. 의식이 사회적 행위의 전제조건이 아니라, 사회적 행위가 의식의 전제조건이다. 그 행위 안에 분리된 요소로서 의식의 개념을 끌어들이지 않고도 사회적 행위의 메커니즘을 추적할 수 있다. 따라서 사회적 행위는 의식의 형태 없이 또는 그와 별개로, 더 기초적 단계 또는 형태로 존재할 수 있다.



제 4장 심리학에서 병행론의 등장

(97)중추신경계에는 경로만 있을 뿐이라는 사실이 명백해졌다
주3) 철학자들 중에 베르그송이 특히 이 점을 강조했다. 그의 '물질과 기억'(Matiere et Memoire)을 참조

(100)내가 말하고 싶은 것은 유기체의 관점에서 심리학 이론에 접근하려면 행동을 강조하는 것, 정적인 측면보다는 역동적인 측면을 강조하는 것이 필연적이라는 점이다.



제 5장 병행론과 '의식'의 양면성

(105-106) 나는 '의식'이라는 용어를 어떤 내용에 대한 접근 가능성을 나타내기 위해 사용하는 것과 그 내용 자체와의 동의어로 사용하는 것을 구분하고 싶다. 눈을 감으면 당신은 어떤 자극들로부터 스스로 차단하는 것이다. 마취제를 취하면 세상은 당신에게 접근가능하지 않다. 이와 유사하게 수면도 사람을 세상에 접근불가능하게 만든다. 나는 이처럼 어떤 영역에 사람이 접근가능하거나 접근불가능하게 만드는 것을 지칭하는 뜻으로 의식이라는 말을 사용하는 경우와 개인의 경험에 의해 결정되는 / 내용 자체를 뜻하는 것으로 사용하는 경우를 구분하고자 한다.

(106-107) 위치가 달라짐에 따라 어떤 지점에 동전이 놓여 있는 거소가 같은 대상에 관해 다른 경험을 하게 된다. 눈의 특성에 따라 달라지는 현상들 또는 과거 경험의 효과에 따라 달라지는 현상들이 있다. 개인들에게는 과거에 일어난 경험에 따라 동전이 달리 경험된다. 한 사람에게는 이렇게 / 보이는 동전이 다른 사람에게는 저렇게 보일 수도 있다. 그러나 동전은 그 자체로서 하나의 실체로 거기에 존재한다. 우리는 개인들에게 공간적 관점에 다라 달리 나타나는 차이를 다룰 수 있기를 원한다. 심리학적 시각에서 훨씬 더 중요한 것은 기억의 관점이다. 기억에 의해 한 사람이 하나의 동전을 보며 다른 사람은 다른 동전을 본다. 이것이 우리가 분리하기를 원하는 특성이다. 우리가 말하는 병행론의 합법성이 있는 곳은 바로 여기다. 즉 물리적으로나 생리적으로 모든 사람에게 공통적으로 결정될 수 있는 대상 그 자체와 어떤 특별한 사람이나 유기체에게 특수한 경험간의 구분에서 병행론의 합법성을 찾을 수 있다는 것이다.

(108) 만약 우리가 이러한 구분을 한계의 밑바닥까지 끌고 간다면 모든 사람에게 동일한 생리적 유기체에 도달하게 되는데, 여기서는 모두에게 동일한 자극들의 집합이 작용한다. 우리는 중추신경계에서 그러한 자극들의 효과를 특별한 개인이 특정 경험을 하는 지점에까지 뒤쫓고 싶어 한다. 특별한 경우에 그렇게 하게 될 때, 우리는 그러한 구분을 일반화하는 기반으로 이 분석을 사용한다. 우리는 한편에는 물리적인 것이 존재하며, 다른 한편에는 정신적인 사건이 존재한다고 말할 수 있다. 우리는 각 사람이 경험하는 세계를 그의 뇌 안에 놓여 있는 인과 연쇄의 결과로 바라볼 수 있다고 가정한다. 우리는 자극을 따라 뇌로 들어가서, 그곳에서 의식이 밝혀져 나온다고 말한다. 이런 식으로 우리는 궁극적으로 모든 경험을 뇌 안에서 찾으려고 하여, 오래된 인식론적 망령이 일어난다.

제 6장 행동주의 프로그램

(109) 어떤 것이 그 개인에게만 접근가능한지, 그 자신의 내적 생활 영역에만 일어나는 것이 무엇인지 하는 것은 그것이 발생하는 상황과의 관련성 속에서 설명되어야 한다. 한 개인은 한 경험을, 또 다른 개인은 또 다른 경험을 가지고 있고, 이 둘은 모두 각자의 생활내력에 의해 설명된다. 그러나 그밖에도 모든 사람의 겨엏ㅁ에 공통되는 것이 있다. 그리고 우리의 과학적 진술은 개인 자신이 경험한 것, 그래서 궁극적으로 자신의 경험만으로 설명될 수 있는 것과 모든 사람에게 속한 경험 사이의 상관관계를 설명한다.

(109-110) 우리의 목표는 그런 특별한 조건에 대응하는 보편적인 용어로 그 경험을 설명하는 것이다. 우리는 할 수 있는 한 그것들을 통제하고 싶어하며, 우리가 그 통제를 실행할 수 있도록 해주는 것은 바로 그 특별한 경험이 발생하는 조건을 결정하는 것이다.

(111) 우리는 어떻게 독특성을 가지고 있는 개인을 보편적인 유형의 반응으로 끌어들일 수 있는가? 그는 다른 사람들과 같은 용어를 사용해야 하며, 같은 단위의 측정을 해야 한다. 그리고 자기 경험을 배경으로 명백한 문화를 수용해야 한다. 그는 자신을 사회구조에 맞추고 그 구조를 자신의 일부르 삼아야 한다. 어떻게 이 모든 것이 이루어질 수 있을까? 우리는 개별적인 개인들을 다루지만 그래도 여전히 이 개인들은 공통된 전체의 일부가 되어야 한다. 우리는 이처럼 공통된 세계와 개인에게 특수한 세계의 상관관계를 얻고자 한다. 그래서 학습 문제, 학교 문제를 공격하며 서로 다른 지능을 분석하려고 하는 심리학을 취하여 가능한 한 공통적인 용어로 이 모든 것을 진술하려고 한다.

(112) 행동주의 심리학이 추구하는 것은 개인의 경험이 발생하는 조건들이 무엇인지를 찾는 것이다. 그런 경험은 우리가 따를 수 있는 행위로 되돌아가게 하는 것이다.

(113) 최근에 흥미를 끌어온 형태주의 심리학(Gestalt Psychology)이라는 또 다른 심리학의 단계를 언급해야 할 필요가 있다. 여기서 우리는 개인의 경험과 그 경험이 일어나는 조건들에 공통적인 경험의요소 또는 단계들이 있음을 인식할 수 있다. 대상 자체에뿐만 아니라 개인의 경험에도 지각의 영역에 어떤 일반적인 형태가 존재한다. 그것들을 찾아낼 수 있다. 

(116) 심리학은 의식을 다루는 것이 아니다. 심리학은 개인의 경험을 그 경험이 진행되는 조건들과 관련시키는 것을 다룬다. 그 조건들이 사회적인 것일 때 사회심리학이 된다. 경험에 접근할 때 행동을 통해 접근하면 행동주의적인 것이다.

제7장 분트와 제스처의 개념

(120) 나는 제스처를 설명하는 방법으로 개들이 싸우는 장면의 예를 들어왔다. 한 마리 개의 제스처는 다른 개의 반응을 나오게 하는 자극 역할을 한다. 그러면 두 마리 개 사이에 관계가 형성된다. 그리고 상대 개의 행위에 대해 처음 개가 또 반응을 하기 때문에, 두 마리 개는 교대로 변화를 겪는다. 한 마리 개가 다른 개를 공격할 준비가 되어 있다는 사실 자체가 다른 개로 하여금 자기 위치나 태도를 변화시키는 원인이 된다. 우리는 여기서 제스처의 대화를 볼 수 있다. 그러나 이것은 의미 있는 제스처는 아니다.

(123) 이제 제스처가 그 이면의 생각을 의미할 때, 그리고 그것이 다른 개인의 마음속에 있는 생각을 불러일으킬 때, 우리는 의미 있는 상징을 갖게 된다. 개가 싸우는 경우에서 우리는 적절한 반응을 불러내는 제스처를 보았다. 그러나 이번에는 우리가 첫 번째 개인의 경험 안에 있는 의미에 대응하고, 또한 두 번째 개인 안에 있는 의미를 불러내는 그런 상징을 갖는다. 제스처가 거기에까지 미치는 경우, 이것을 우리는 '언어'라고 하게 된다. 여기서 제스처는 '의미 있는 상징'으로서 어떤 의미를 부여받게 되는 것이다.

(124) 그리고 의미 있는 제스처 또는 의미 있는 상징은 의미를 부여받지 못한 제스처보다 그러한 적응에 훨씬 더 큰 혜택을 주는데, 그 이유는 의미 있는 제스처나 상징이 그것을 내보내는 개인이나 그와 함께 주어진 사회적 행위 안에서 참가하는 다른 개인들에게서 그 제스처(또는 그 의미)에 대한 똑같은 태도를 끌어내고, 이로써 그 제스처에 대한 태도를 그가 (자기 행동의 구성요소로서) 의식할 수 있도록 만들어서 그 태도에 비추어 그들에게 나중의 행동을 적응할 수 있게 하기 때문이다. 간단히 말해서, 의식적 또는 의미 있는 제스처의 대화는 비의식적 또는 의미 없는 제스처의 대화보다 사회적 행위-그 제스처를 하는 개인들 각자가 자기에 대한 상대방의 태도를 알아차리는 것-안에서 상호 적응하는데 훨씬 더 적절하고 효과적인 메커니즘이다.

(125) 단지 의미 있는 상징으로서의 제스처 측면에서만 마음이나 지능의 존재가 가능하다. 단지 의미 있는 상징으로서의 제스처일 때에만 생각 - 이런 제스처를 통해 개인이 자기 자신과 내면화된 또는 암묵적인 대화를 하는 것이라고 간단히 정의됨 -이 일어난다. 우리가 사회적 과정에서 다른 개인들과 함께 진행하는 제스처의 외적인 대화를 경험함으로써 내면화되는 것이 생각의 핵심이다. 제스처는 이렇게 내면화되어서 의미 있는 상징이 되는데, 그 이유는 주어진 사회 또는 사회적 집단의 모든 개개 구성원들에게 같은 의미를 지니게 되기 때문이다. 즉 의미 있는 상징이 된 제스처는 그 제스처를 하는 사람과 거기에 반응하는 사람에게 똑같은 태도를 일으키기 때문이다. 그렇지 않으면 개인은 그것을 내면화할 수도 없고 그것을 의식하여 의미를 파악할 수도 없을 것이다.

(128) 분트는 이처럼 커뮤니케이션이 우리가 '정신'이라고 하는 것의본질을 이루는 기초가 된다는 중요한 사실을 간과했다. 이 사실을 인정할 때 비로소 정신을 행동주의적으로 설명하는 데 가치와 이점이 있다는 것을 정확히 깨닫게 된다. 그러므로 분트의 커뮤니케이션 분석은 커뮤니케이션할 수 있는 정신의 존재를 미리 가정하며, 이 존재는 그의 심리학적 기반 위에서 설명할 수 없는 미스터리로 남아 있다. 반면에 커뮤니케이션의 행동주의적 분석에는 그런 전제가 없고, 그 대신 커뮤니케이션과 사회적 경험의 측면에서 정신의 존재를 설명하거나 해석한다.

제 8장 모방과 언어의 기원

(137) 그러나 사람들 사이에서는 특히 음성 제스처를 만들어낼 때 모방하는 경향이 있는 것으로 보인다. 우리는 이런 경향을 사람에게서뿐만 아니라 새들에게서도 찾을 수 있다. 만약 당신이 독특한 방언을 사용하는 지역에 가서 오랫동안 지낸다면, 당신은 같은 방언을 말하게 된다. 하지만 이것은 당신이 원하던 것이 아닐 수도 있다. 가장 간단히 말하면 당신은 무의식적으로 모방하는 것이다. 다른 다양한 행동양식의 경우도 마찬가지다. 만약 당신이 어떤 사람을 생각하면 당신은 자신이 그 사람이 말하는 방식대로 말하게 되는 것을 발견할 수 있다. 한 개인이 가지고 있는 행동양식은 그 사람이 당신의 마음속에 들어올 때 당신 스스로 실행하는 경향을 보이는 것이다. 그것이 우리가 '모방'이라고 부르는 것으로, 실제로 하등동물에게서는 그런 행동을 볼 수 없다는 점이 흥미롭다...(중략) 그렇지만 일반적으로 다른 개체들의 과정을 수용하는 것은 하등동물에게는 자연스러운 것이 아니다. 모방은 사람에게만 속하는 것으로, 사람만이 독립적인 의식적 경지에까지 이를 수 있다.

(137-138) 유기체의 입장에서 단순히 다른 유기체의 행동을 보거나 듣는 것을 / 재생산해내는 경향으로서의 모방은 기계적으로 불가능하다. 유기체에게 도달하는 모든 시각적, 청각적 자극이 경험에서 보고 들은 것을 재생산해내게 할 정도로 잘 정비되어 있는 유기체는 상상할 수 없다.

제 9장 음성 제스처와 의미 있는 상징

(144) 다른 사람이 느끼게 하기 위해서 보이는 수단으로 신체 표현을 사용하는 사람은 배우뿐이다. 그는 거울을 끊임없이 사용함으로써 자기가 어떻게 보일지를 스스로에게 보여주는 반응을 취한다. 그는 분노, 사랑, 이것, 저것 또는 다른 태도를 새기며, 거울을 통해 스스로 자기가 어떻게 그렇게 보일 수 있는지를 알아본다. 나중에 그가 제스처를 사용할 때 이것은 정신적 이미지로서 존재한다. 그는 그 특별한 표현이 공포를 불러일으킨다는 사실을 인식한다. 만약 우리가 음성 제스처를 배제한다면, 그것은 단지 다른 사람이 반응하는 것처럼 스스로 제스처에 반응하는 위치에 도달하려고 거울을 사용할 때만 그렇다. 그러나 음성 제스처는 다른 사람이 대응하는 것처럼 스스로의 자극에 대응하는 능력을 부여하는 것이다. 

제 10장 생각, 커뮤니케이션, 의미 있는 상징

(148) 우리는 특히 음성 제스처를 사용해서 우리가 다른 사람들에게서 불러내는 반응을 계속해서 스스로에게서도 불러내며, 달느 사람들이 우리 행동에 대해 지니는 태도를 우리 스스로도 취하게 된다.

(148) 왓슨과 같은 행동주의자는 우리의 모든 사고과정이 음성화되었다고 주장한다. 생각할 때 우리는 단순히 어떤 단어를 사용하기 시작한다. 그것은 어떤 의미에서 사실이다. 그러나 왓슨은 여기에 포함되는 모든 것을 고려하지 않았다. 즉 이 자극이 정교한 사회적 과정에서 본질적인 요소가 되어 사회적 과정에 가치를 부여한다는 사실을 간과한 것이다. 그러한 음성적 과정은 이처럼 아주 중요하며, 음성적 과정은 그것과 어울리는 지능, 사고과정과 함께 단순히 서로에 대해 특별한 음성적 요소를 실행하는 것만이 아니다. 이런 관점은 언어의 사회적 배경을 무시한 것이다.

(152) 우리는 동물들이 생각하지 않는다고 말한다. 동물은 자기가 책임져야 할 위치에 자기 자신을 놓지 않는다. 동물은 다른 쪽 상대방의 입장에 자기를 놓고 그 결과로서 "그는 그런 식으로 행동할 것이고, 그러면 나는 이런 식으로 행동할 것이다"라고 생각하지 않는다. 

제 11장 의미

(158) 이와 유사하게, 사회적 과정은 어떤 의미에서 그것이 반응하는, 또는 적응하는 대상들을 구성한다. 다시 말해서, 대상들은 사회적 과정에서 서로에게 적응하며 반응하거나 행동하는 다양한 개인 유기체들을 통해 경험과 행동의 사회적 과정의 이른 진화 단계에서 제스처의 대화 형태, 그리고 진화의 나중 단계에서는 언어의 형태를 취하는 커뮤니케이션 수단을 통해 가능해진다. 

(162) 의미의 의미 문제에 관해 많이 민감하게 다루어왔다. 이 문제를 해결하기 위한 시도에서 정신적 상태에 의지할 필요는 없다. 우리가 보아왔듯이, 의미의 본질은 사회적 행위의 구조 안에 내재되어 있다는 사실이 밝혀져왔기 때문이다. 구체적으로, 그 의미는 3종의 기본적인 개개 요소들 사이의 관계 안에 내재되어 있는데, 즉 한 개체의 제스처, 그 제스처에 대한 두 번째 개체의 반응, 그리고 첫 번째 개체의 제스처에 의해 시작된 해당 사회적 행위의 완성이 3종의 기본 요소다. 따라서 의미의 본질이 사회적 행위의 구조 속에 내재되어 있다는 사실 때문에 사회심리학 안에서 필연성을 추가로 강조해야 할 필요가 있는데, 그것은 바로 어떤 개인들의 집단이 포함되어 진행 중인 경험과 행동의 사회적 과정을 처음부터 가정하고 시작해야 한다는 것이다. 그 바탕 위에서 개인들의 정신, 자아 그리고 자기의식의 존재와 발달이 가능해진다.

제 16장 정신과 상징

(205-206) 우리는 한 세트의 상징을 가지고 있어서, 그것을 통해 어떤 특성들을 지시하며, 그 특성들을 지시하면서 그것들을 즉각적인 환경에서 분리해내고, 단순히 하나의 관계를 명확히 유지한다. 우리는 곰의 발자국을 분리하여 그 발자국을 만든 동물과의 관계에만 유지한다. 우리는 그것에 반응하는 것이지, 다른 데 반응하는 것이 아니다. 우리는 곰을 지시하는 것으로서 그것에 의미를 부여하며, 피하거나 사냥해야 할 것으로 경험 안에서 그 대상이 지니는 가치를 부여한다. 대상에 대한, 그리고 그 대상에 속하는 반응에 대한 관계 안에서 이와 같은 중요한 특성들을 분리하는 능력은 우리가 일반적으로 인간이 사물에 관해 생각한다고, 또는 정신을 가지고 있다고 말할 때 의미하는 것이다. 그러한 능력은 상징이라는 수단을 통한 인간의 사고과정을 흰 쥐의 경우에서 반사의 조건화 / 와 아주 큰 차이를 가져오게 한다. 

(206. 주2) 사물이나 대상의 의미는 그것들의 실제 내재적인 특성 또는 질을 뜻한다. 어떤 주어진 의미가 놓이는 위치는 우리가 '가지고 있다'고 말하는 것 안에 존재한다. 상징을 사용할 때 우리는 사물의 의미를 지칭한다. 상징은 의미를 지니는 사물이나 대상의 의미를 나타낸다. 상징은 어떤 것들이 존재하는 (또는 즉각적으로 경험되는) 상황이나 시점 안에서, 직접적으로 존재하지 않는 다른 경험의 부분을 지적하거나 지시하거나 나타내는 경험의 부분들로서 주어진다. 상징은 따라서 단순한 자극 대치물 이상의 것, 즉 조건화된 반응이나 반사에 대한 단순한 자극 이상의 것이다. 조건반사, 즉 단순한 자극 대치물에 대한 반응은 의식을 포함하지도 않고 포함할 필요도 없다. 반면에 상징에 대한 반응은 의식을 포함하며 또 포함해야 한다. 조건반사와 태도, 의미의 의식을 합치면 언어를 구성하는 것이 된다. 따라서 그것이 사고와 지적 행동의 기반 또는 메커니즘을 형성한다. 언어는 개인이 대상에 어떻게 반응할 것인지를 다른 개인과 함께 서로에게 지시해주는 수단이며, 따라서 대상의 의미가 무엇인지를 커뮤니케이션하는 수단이다. 언어는 단순한 조건반사 체계가 아니다. 이성적인 행위에는 항상 자기에 대한 반사적 저항, 즉 다른 개인들에게 자신의 행위나 제스처들이 지니는 의미를 스스로에게도 지시해주는 부분이 포함된다. 그리고 그러한 행위의 경험적 또는 행동적 기반, 즉 사고과정의 신경생리학적 메커니즘은 우리가 보아왔듯이 중추신경계 안에서 찾을 수 있다.

상징은 단순한 자극 대치물 이상의 것이다.

(208-209) 하등동물의 경우에는 그런 상황에 대한 증거가 없다. 어떤 동물/행동에서도 상징, 커뮤니케이션 방법, 다른 반응들에 대응하는 것을 상세히 실현할 만한 것을 찾을 수 없으며, 따라서 개체의 경험 안에 모든 가능성을 지니고 있을 수 없다는 사실에서 이것은 분명해진다. 반사적 지능을 가진 존재를 하등동물의 행위에 구분해주는 것은 바로 이 부분이다. 그리고 그것을 가능하게 해주는 메커니즘은 언어다. 우리는 언어가 행위의 일부임을 인식해야 한다. 그러나 정신에는 사물의 특성에 대한 관계가 포함된다. 그 특성들은 사물 속에 있으며, 반면에 자극은 어떤 의미에서 유기체 속에 존재하는 반응, 밖에 존재하는 사물에 대한 반응을 불러일으킨다. 모든 과정이 다 정신적 산물인 것은 아니며, 따라서 뇌 안에서만 원인을 찾을 수 있는 것도 아니다. 정신성은 바로 상징들의 집합이 매개하는 상황에 유기체가 관련을 맺을 때 발생한다.

제 17장 정신이 반응, 환경과 맺는 관계

\newpage
\subsection{정신, 자아 그리고 사회 18장 220816}
(223) 자아는 생리적 유기체의 특성과는 다른 특성을 지닌다. 자아는 발달 과정을 지니는 것이다. 자아는 태어나자마자 처음부터 이미 존재하는 것이 아니라, 사회적 경험과 활동 과정에서 등장하는 것이다. 즉 자아는 사회적 과정 전체에서, 그리고 사회적 과정 속에서 다른 개인들과 이루어가는 관계의 결과로, 특정 개인 안에서 발달해가는 것이다. \\
(224) 우리는 자아와 신체의 차이를 명확히 구분할 수 있다. 신체는 거기에 존재할 수 있으며, 경험 안에 자아가 포함되지 않아도 아주 지능적인 방식으로 작동할 수 있다. 자아는 스스로에게 대상이 되는 특성을 지니며, 그 특성이 자아를 다른 대상과 구분해주면서 동시에 신체와 구분해준다. \\
(225) 내가 끌어내고 싶은 것은 대상으로서의 자아가 그 자체에 대해 지니는 특성이다. 이 특성은 반사적인 '자아'로 나타나며, 주관적이면서 동시에 객관적일 수 있는 것을 말한다. \\
(225-226) 자아는 사물에 비추어 자신의 몸의 일부를 포함한 사물과 행위에 둘러싸여 있는 유기체와는 완전히 구분된다. 후자는 다..른 대상들과 같은 대상이 될 수 있지만, 저기 장 안에 존재하는 대상일 뿐이며, 유기체의 대상이 되는 자아를 포함하지 않는다. 이것이 흔히 간과되는 부분이라고 생각한다. 동물의 생활을 의인화하여 재구성하는 것이 잘못되는 것은 바로 이런 사실 때문이다. 한 개인이 어떻게 해서 (경험적으로) 스스로에게 대상이 되는 방식으로 자기 바깥에 있을 수 있는가? 이것은 자아 또는 자기의식의 본질적인 심리학적 문제다. 그 해결점은 특정 사람이나 개인을 시사하는 사회적 행동이나 행위의 과정을 지칭함으로써 찾을 수 있다. \\
(227) '커뮤니케이션'이라는 말은 그것이 유기체나 개인이 스스로에게 대상이 될 수 있는 행동의 형태를 제공한다는 사실에서 중요성을 지닌다. 우리가 지금까지 논의해온 것은 그런 종류의 커뮤니케이션이다. 병아리에게 하는 암탉의 꼬꼬댁 소리, 또는 늑대 떼를 향한 늑대의 울부짖음, 또는 소 우는 소리와 같은 의미에서의 커뮤니케이션이 아니라, 의미 있는 상징이라는 의미에서의 커뮤니케이션, 즉 다른 사람을 향해서뿐만 아니라 개인 스스로에게도 향하는 커뮤니케이션인 것이다. 이런 유형의 커뮤니케이션이 행동의 일부가 되는 한, 그것은 적어도 자아를 도입하는 행동의 일부가 된다. \\
(227-228) 그러한 자아는 주로 생리적인 유기체가 아니라고 말하고 싶다. 생리적인 유기체가 자아에 꼭 필요하기는 하지만, 생리적인 유기체 없이도 .. 우리는 자아를 적어도 '생각할' 수는 있다.\\

(228) 스스로에게 대상이 될 수 있는 자아는 본질적으로 사회적 구조이며, 사회적 경험 안에서 일어난다. 자아가 일어난 후에는 어떤 의미에서 그 자체로서 사회적 경험을 제공하며, 따라서 우리는` 절대적으로 혼자 있는 자아를 상상할 수 있다. 그러나 자아가 사회적 경험 바깥에서 일어나는 것을 상상하기는 불가능하다. \\

(229) 우리는 우리가 말하는 것을 이해함으로써 다른 사람들에게 향하는 자신의 언급을 계속적으로 따라가며, 이러한 이해를 통해 계속 이야기하는 방향을 인식한다. 우리는 현재의 말과 행동을 통해 앞으로 우리가 말하려는 것, 우리가 하려는 것을 찾으며, 이 과정에서 우리는 지속적으로 그 과정 자체를 통제한다. 제스처의 대화에서는 다른 사람에게서 어떤 반응을 불러일으키고 그 후에 스스로의 행동을 변화시킨다. 그리하여 우리가 하려고 시작했던 것이 다른 사람의 반응으로 변화하기도 한다.  \\

(230-231) 우리는 일상적인 행동과 경험에서 개인이 자기가 하고 말하는 것의 상당 부분을 의미하지 않는다는 사실을 깨닫는다. 우리는 흔히 그런 개인은 자신이 아니라고 말한다. 우리는 중요한 것을 두고 왔다는 것을 깨닫고 인터뷰에서 황급히 나올 수도 있으며, 자아의 일부는 말한 것에 들...어가지 않을 수도 있다. 커뮤니케이션에 들어가는 자아의 양을 결정하는 것은 사회적 경험 자체다. 물론, 자아의 상당 부분은 꼭 표현되어야 할 필요가 있는 것은 아니다. 우리는 다른 사람들에 대한 서로 다른 관계들의 전체 시리즈를 실천한다. 우리는 이 사람에게는 이것이 되고, 저 사람에게는 저것이 된다는 것을 안다. \\

(232) 우리가 여기서 처해 있는 상황은 서로 다른 여러 개의 자아가 있을 수 있으며, 그것은 우리가 앞으로 될 수 있는 자아가 무엇인지에 관련되어 있는 사회적 반응들의 집합에 따라 달라진다는 사실이다. 

(232) 완전한 자아의 단위와 구조는 하나의 전체로서 사회적 과정의 단위와 구조를 반영한다. 그리고 자아를 구성하는 기본적인 자아 각각은 개인이 함축되어 있는 그 과정의 다양한 측면들 중 하나의 단위와 구조를 반영한다. 달리 말하면, 완전한 자아를 구성하고 조직하는 다양한 기본적인 자아들은 하나의 전체로서 사회적 과정이 지니고 있는 다양한 측면의 구조에 대응하는 완전한 자아 구조의 여러 측면들이다. 따라서 완전한 자아 구조는 완전한 사회적 과정을 반영한다. 사회적 집단의 조직화와 통합은 그 집단이 관여하거나 실행하는 사회적 과정 안에서 발생하는 자아들 중 어느 하나의 조직이나 통합과 동일하다. 
주5. 정신의 단위는 자아의 단위와 동일하지 않다. 자아의 단위는 사회적 행동의 완전한 관계적 패턴의 단위, 그리고 개인을 암시하는 경험과 자아의 구조 안에 반영되어 있는 경험으로 구성된다. 그러나 이 완전한 패턴의 측면들 또는 세부 특징들 중 많은 부분이 의식 속으로 편입되지 않는다. 따라서 정신의 단위는 어떤 의미에서 더 포괄적인 자아의 단위에서 추상화된 것이라고 할 수 있다. 


\newpage
\subsection{정신, 자아 그리고 사회 19장 220817}
(234) 그렇다면 여기서 우리는 개인이 적어도 자기 자신 안에서 반응을 불러일으킬 수 있어서 이 반응들에 대응할 수 있는 상황, 즉 상대방에게 갖는 것과 똑같은 효과를 개인에게도 갖는 사회적 자극이 존재하는 조건을 접하게 된다. 예컨대 그것이 바로 언어 속에 함축되어 있는 것이다. 그렇지 않으면, 개인이 자기가 말하는 것의 의미를 얻지 못하기 때문에 의미 있는 상징으로서의 언어는 사라질 것이다.

(235) 우리 인간의 사회적 환경이 가지고 있는 특성은 사회적 활동이 지니는 특성 덕분에 존재 가치가 생기는 것이다. 그리고 우리가 보아 왔듯이 그 특성은 커뮤니케이션 과정 안에서 발견되며, 더 특수하게는 의미의 존재의 바탕이 되는 3자 관계 안에서 발견되는 것이다. 한 유기체의 제스처와 적응적 반응의 관계는 다른 유기체에 의해 판단된다. 즉 그 유기체가 시작한 행위의 결과 또는 완성을 지시하는 능력을 지칭하는 것이며, 따라서 제스처의 의미는 두 번째 유기체가 제스처로서 그것에 반응하는 것이다. 

(235) 우리는 때로 마치 한 사람이 마음속에 전체 주장을 완전하게 구축할 수 있는 것처럼 이야기하며, 따라서 그것을 다른 사람에게 완전히 말로 전달할 수 있다고 믿는다. 실제로 우리의 사고과정은 항상 상징이라는 수단을 통해 일어난다. 상징이 없어도 경험 속에서 '의자'의 의미를 알 수는 있지만, 그 경우는 의자에 관해 생각하는 것이 아니다. 우리가 하는 것에 관해 생각하지 않고, 즉 의자에 다가가는 것이 필경 우리 경험 속에 이미 일어나 있어서 의미가 존재하는 상태에서 의자에 앉을 수도 있다. 그러나 만약 의자에 관한 생각을 하면 반드시 그에 대한 어떤 종류의 상징을 갖게 된다. 그것은 의자의 형태일 수도 있고, 누군가 앉아있는 태도가 될 수도 있지만, 이 반응을 일으키는 언어 상징이 될 확률이 가장 높다.

(236) 우리의 상징은 보편적인 것이다. 절대적으로 특수한 것은 아무것도 이야기할 수 없다. 어떤 의미를 가지고 있다고 당신이 말하는 것은 무엇이든지 보편적인 것이다.

(238) 우리가 말하는 것의 상당 부분은 진정으로 미적 특성을 지니고 있지 않다. 대부분의 말에서 우리는 우리가 의도적으로 불러일으키는 감정을 느낀다. 우리는 대개 우리가 다른 사람 안에서 불러일으키는 감정적 반응을 우리 스스로의 마음속에서도 불러일으키는 언어 자극을 사용하지 않는다. 물론, 사람은 감정적 상황 속에서 공감을 느낀다. 하지만 사람이 그 안에서 찾는 것은 결국 자기 경험 안에서 개인을 지지하는 것이 다른 사람 안에서도 존재하는 그런 것이다. 

(239) 커뮤니케이션에 필수적인 것은 그것이 다른 개인에게서 불러일으키는 것을 스스로에게서도 불러일으켜야 한다는 것이다. 상징은 스스로 같은 상황에 있다는 것을 알게 되는 누구에게나 그런 종류의 보편성을 지니고 있어야 한다. 

(239) 헬렌 켈러 같은 시각장애인의 경우 스스로에게 주어지는 것과 같이 다른 사람에게 주어질 수 있는 것은 접촉 경험이다. 헬렌 켈러의 마음이 구성되는 것은 바로 그런 종류의 언어를 통해서다. 그녀가 깨달았듯이, 그녀가 정신적 내용, 또는 자기(self)라고 부르는 것을 얻을 수 있었던 것은 다른 사람에게서 불러일으키는 반응들을 스스로에게서도 불러일으킬 수 있는 상징을 통해 다른 사람과 커뮤니케이션할 수 있었을 때 비로소 가능했다. 




\newpage
\subsection{과학학 220812}
과학. 관찰(보는 것). objekt화. 사회에서의 경제, 정치, 심리를 객관화하는 것이 가능? 숲 안에서 숲을 보는 것 \\

지식의 축적과 학문의 size가 커져서 과학학이 나온 것이면, 공학학은? 과학과 공학은? \\


\newpage
\subsection{교보문고 220812}

종교생활의 원초적 형태. 한길사 \\
(154) 초자연적인 어떤 것에 대해 말할 수 있기 위해서는 \emph{사물들의 자연적인 질서}가 존재한다는 느낌, 다시 말해 우주현상들이 법칙이라고 불리는 필연적인 관계에 의해 서로 연결되어 있다는 느낌을 미리 가지고 있어야만 한다.  \\
(155) 일단 이러한 법칙이 받아들여지고 나면 이 법칙에 위배되는 모든 것은 필연적으로 자연을 벗어난 것으로, 또한 이성을 벗어난 것으로 보인다. 이러한 의미에서 볼 때, 자연적인 것이란 합리적인 것이다. \\
(157) 무엇보다도 초자연적인 것을 예기치 못한 뜻밖의 일이라고 결론지어서는 안 된다. 새로운 것은 자연과 대립되기도 하지만 자연의 일부가 될 수도 있다. 망냑 우리가 일반적으로 결정된 순서에 따라 현상들이 연속된다는 것을 증명한다고 해도, 우리는 이러한 질서 역시 대략적인 것에 불과하고, 다른 것과 정확하게 일치하지 않고, 때로는 자기 자신과도 일치하지 않으며 온갖 종류의 예외들을 포함하고 있다는 사실을 알고 있다. 만일 우리가 조금이라도 경험이 있다면 우리의 기대가 번번히 좌절당하는 것에 익숙해져 있고, 기대에 어긋나는 일이 자주 일어나기 때문에 이러한 것들이 특이하게 보이지도 않을 것이다. 어떤 우연성은 일관성과 마찬가지로 경험의 자료이다. 따라서 둘 중 어느 하나가 나타나게 된 원인이나 힘이 다른 것이 나타나게 된 원인이나 힘과 전혀 다르다고 말할 만한 하등의 이유가 없다. 따라서 초자연성에 대한 관념을 갖기 위해서는 예기치 못한 사건들을 목격하는 것으로는 불충분하다. 즉 예기치 못한 사건들이 불가능한 사건으로 인식되어야 한다. 다시 말해서 그러한 사건들이 사물들의 본질 속에 필연적으로 포함된 것으로 보이는 질서-이러한 질서가 옳을 수도 있고 그를 수도 있겠지만-와 화합될 수 없는 것으로 여겨져야 한다. \\

(171) 모든 알려진 종교적 믿음은 그것이 단순하거나 복잡하거나 간에 똑같은 공통의 특성을 보여준다. 이러한 믿음은 실제적이거나 관념적이거나 인간이 생각하는 모든 사물들의 분류를 전제로 한다. 즉 분명한 두 요엉, \emph{속된 것}과 \emph{거룩한 것}이라는 용어로 잘 표현해주는바, 일반적으로 지칭되는 두 부류 또는 서로 반대되는 두 장르로 분류하는 것을 전제로 하고 있다.

(188) 따라서 우리는 다음과 같은 정의를 내리게 된다. \emph{종교란 성스러운 사물들, 즉 구별되고 금지된 사물들과 관련된 믿음과 의례가 결합된 체계다. 이러한 믿음과 의례들은 교회라고 불리는 단일한 도덕적 공동체 안으로 그것을 신봉하는 모든 사람을 통합시킨다.}
\newpage
\subsection{미드 16장 220810}
제 16장 정신과 상징

(205-206) 우리는 한 세트의 상징을 가지고 있어서, 그것을 통해 어떤 특성들을 지시하며, 그 특성들을 지시하면서 그것들을 즉각적인 환경에서 분리해내고, 단순히 하나의 관계를 명확히 유지한다. 우리는 곰의 발자국을 분리하여 그 발자국을 만든 동물과의 관계에만 유지한다. 우리는 그것에 반응하는 것이지, 다른 데 반응하는 것이 아니다. 우리는 곰을 지시하는 것으로서 그것에 의미를 부여하며, 피하거나 사냥해야 할 것으로 경험 안에서 그 대상이 지니는 가치를 부여한다. 대상에 대한, 그리고 그 대상에 속하는 반응에 대한 관계 안에서 이와 같은 중요한 특성들을 분리하는 능력은 우리가 일반적으로 인간이 사물에 관해 생각한다고, 또는 정신을 가지고 있다고 말할 때 의미하는 것이다. 그러한 능력은 상징이라는 수단을 통한 인간의 사고과정을 흰 쥐의 경우에서 반사의 조건화 / 와 아주 큰 차이를 가져오게 한다.
(206. 주2) 사물이나 대상의 의미는 그것들의 실제 내재적인 특성 또는 질을 뜻한다. 어떤 주어진 의미가 놓이는 위치는 우리가 '가지고 있다'고 말하는 것 안에 존재한다. 상징을 사용할 때 우리는 사물의 의미를 지칭한다. 상징은 의미를 지니는 사물이나 대상의 의미를 나타낸다. 상징은 어떤 것들이 존재하는 (또는 즉각적으로 경험되는) 상황이나 시점 안에서, 직접적으로 존재하지 않는 다른 경험의 부분을 지적하거나 지시하거나 나타내는 경험의 부분들로서 주어진다. 상징은 따라서 단순한 자극 대치물 이상의 것, 즉 조건화된 반응이나 반사에 대한 단순한 자극 이상의 것이다. 조건반사, 즉 단순한 자극 대치물에 대한 반응은 의식을 포함하지도 않고 포함할 필요도 없다. 반면에 상징에 대한 반응은 의식을 포함하며 또 포함해야 한다. 조건반사와 태도, 의미의 의식을 합치면 언어를 구성하는 것이 된다. 따라서 그것이 사고와 지적 행동의 기반 또는 메커니즘을 형성한다. 언어는 개인이 대상에 어떻게 반응할 것인지를 다른 개인과 함께 서로에게 지시해주는 수단이며, 따라서 대상의 의미가 무엇인지를 커뮤니케이션하는 수단이다. 언어는 단순한 조건반사 체계가 아니다. 이성적인 행위에는 항상 자기에 대한 반사적 저항, 즉 다른 개인들에게 자신의 행위나 제스처들이 지니는 의미를 스스로에게도 지시해주는 부분이 포함된다. 그리고 그러한 행위의 경험적 또는 행동적 기반, 즉 사고과정의 신경생리학적 메커니즘은 우리가 보아왔듯이 중추신경계 안에서 찾을 수 있다.

상징은 단순한 자극 대치물 이상의 것이다.

(208-209) 하등동물의 경우에는 그런 상황에 대한 증거가 없다. 어떤 동물/행동에서도 상징, 커뮤니케이션 방법, 다른 반응들에 대응하는 것을 상세히 실현할 만한 것을 찾을 수 없으며, 따라서 개체의 경험 안에 모든 가능성을 지니고 있을 수 없다는 사실에서 이것은 분명해진다. 반사적 지능을 가진 존재를 하등동물의 행위에 구분해주는 것은 바로 이 부분이다. 그리고 그것을 가능하게 해주는 메커니즘은 언어다. 우리는 언어가 행위의 일부임을 인식해야 한다. 그러나 정신에는 사물의 특성에 대한 관계가 포함된다. 그 특성들은 사물 속에 있으며, 반면에 자극은 어떤 의미에서 유기체 속에 존재하는 반응, 밖에 존재하는 사물에 대한 반응을 불러일으킨다. 모든 과정이 다 정신적 산물인 것은 아니며, 따라서 뇌 안에서만 원인을 찾을 수 있는 것도 아니다. 정신성은 바로 상징들의 집합이 매개하는 상황에 유기체가 관련을 맺을 때 발생한다.


\newpage
\subsection{부천갔다온일기 220808}
\subsubsection{1장. 자본의 개념}
Q. 사회적 자본에서 사회적과 자본의 양립이 어떻게 이루어지는지?

자본의 측정.
경제학에서의 유량과 저량 개념. 유량은 기간에서. 저량은 시점에서.
자본은 생산수단(means of production), 실물자본(real capital), 화폐자본(money capital)...

자본을 측정하는 순간에는 고정된 것이지만, 자본은 축적(accumulate)되는 것이며 유동적.

우회생산. round about production (Austria)
-> 추후 찾아보는 것으로.

예를 들어, 물고리를 잡는다면 손에서 낚시, 낚시에서 그물, 그물에서 배와 그물, 그리고 GPS... 이런식으로 생산수단의 복잡화와 함께 생산성이 더 증가하는데, 이전의 것을 더 잘하게 되고 할 수 없는 것을 가능하게 하는 것. 이러한 고도화하는것을 가능하게 하는 의미로서 '우회'(round about? round-over?)

어쨌든 capital은 축적, 변화하는 것. 즉 동적인 것. 목적을 가지고 축적

질문못한 부분) 사람들은 사회적 자본을 축적하기 위해서 사람들을 만나거나 그러지는 않음. 앞의 예시든 다른 것들이든 자본을 축적해서 이익을 추구하는 것이 아니라, 사회적 자본은 어떻게 조작적으로 정의된? 그런 개념인 것 같은데?

\subsubsection{2장. 루만의 기능분석}

루만의 경우 자신은 기능분석(functional analysis. 독어로 이야기를 많이 해서 나중에 보충)

자신이 생각하는(그리고 맞다고 보는) 루만에게서 중요한 것은 기능. 그런데 그냥 기능이 있는 것이 아니라,

문제(problem)가 있고 이것을 푸는 것으로서의 기능이 있고, 그 기능을 실행하는 system이 있다. 1984 soziale systems. 여기는 다 루만이에요. 시스템이 있으면 문제를 푸는 기능을 하는 것이고 문제가 있다는 것을 의미. 

이 문제에 대한 깊은 천착을 통해, 기존의 표현의 낯설어질때까지 천착을 하면. 그것이 당연하지 않은 것이 되고 기존의 것이 낯설게 된다. 그렇게 되었을 때 기능적 등가물(functional equivalant)가 만들어질 수 있는 것(문제를 푸는 거소가 같은 등가물이 만들어지는것. 기존의 방법이 여러 개 중 하나가 될 때까지). 반면에 천착하지 않으면 기존문제를 수정, 개선하는 수준으로서 path-dependency와 같은 것이 된다. 기능적 등가물의 생성은 곧 경로의 변경을 의미하여 이것이 역설로 이어지 것.

(예를 들어, 민주주의가 피지배자의 지배임을 뜻하는, 그냥 아 역설이구나 하는 것은 중간에 멈추는 것.) 거기에 대한 깊은 천착이 부재. 역설을 전개(unfold)해서 여러 기능적 등가물을 만들어내는 것. 민주주의처럼 혹은 박근혜 탄핵처럼 그러한 역설이 현실에서 일어나는 것. 

헤겔의 주인과 노예의 변증법(노예는 생존을 위해서 더 많은 것을 하고 의식이 복잡해서 노예의식의 우월성을 이야기하는 것)

\subsubsection{3장. 이중의 우연성}
double-contingency
의미적으로는 사실은 double double-contingency에 가까움. 이 이중의 우연성은 파슨스가 먼저 제시한 것\footnote{아마 기억이 맞다면 https://www.amazon.com/Toward-General-Theory-Action-Theoretical/dp/0765807181 이 책인 것으로 기억.}

ego와 alter ego가 있는데 서로가 서로를 조건 짓는(이중의 우연성) 것에서 더 나아가 내가 그러한 것 + 상대방 역시 그러한 것이라는 점에서 double double contingency가 됨. \\

ego는 alter ego의 strategy에 depend on한다. 그래서 쟤가 뭘 좋다고 할지에 대해서 생각을 하는게. 걔가 제일 좋다고 생각하는게 내가 제일 좋은 거에 의존을 받는 것(이중). 그런데 쟤(alter ego)도 똑같이 그렇게 생각하기 때문에 double double contingency가 되는 것.

사회 시스템이 풀려고 하는 문제를 double-contingency problem으로. 문제를 깊게 풀려고 함.\footnote{https://en.wikipedia.org/wiki/Robert_Axelrod}\footnote{The Evolution of Cooperation}\footnote{ch 2. the success of TIT for TAT in computer tournaments 를 보고 책을 찾았는데 아마 저 책인 것 같음. 책 훑어보다가 죄수의 딜레마도 있는 것 같은데 책 좀 찾아볼게요}

일부내용: what the other player is likely to be doing . Further, what the other is likely to be doing may well depend on what the player expects you to do

루만의 문제를 나의 문제로 받아들이는 것. 그 사람이 어떻게 했는지를 보는 것이 아니라 그 문제가 나의 문제가 되어야 하는 것.

order from noise. 못알아듣는게 많아지는데 그러면 어떻게든 알아들으려고함. 이 noise는 문제와 같은말. auto-카탈리자투아(catalyst와 유사한 의미같은데 정확히 영어로 뭐랑 대응되는지 모르겠음). 스스로 문제를 해결하려 하는 것.

\subsubsection{3장 부록. 희소성}
사회에서 희소성이 어떻게 생기는가. here가 있고 there가 있고, 그걸 구분짓는 boundary가 있음. 사실은 그 boundary까지 생각하면 세 개로 구분되는 것이라고 할 수도 있지만... Law of form. two-side를 하려면 경계가 있어야 하고, 결국 그거는 세 개를 말함.

물건이 있고, 다른 사람들이 그 물건을 부족해하는데 한 사람이 그걸 점유해버리면 그 사람의 희소성은 감소하지만 사회적으로 다른 사람들의 희소성은 증가를 하는 것(이전의 가능성을 없애버리는 의미). 희소성을 감소시키려니까 희소성이 증가됨. 이 역설을 전개해서, 그 전개를 거의 완전하게 한 것이 화폐. 화폐를 가지고 점유. 화폐를 준다는 것은 지불가능성을 준다는 것이고, 화폐는 희소성을 감소시키는 것으로서. 희소성이 역설이 보이지 않음.

의식과 몸의 관계. 의식 시스템이 몸을 괴롭히고 몸은 고통을 통해서 신호를 주는 것.

공부. 눈사람을 만들 때 눈을 쌓아가는 과정과도 같음. accumulation은 일단 붙을 게 있어야 가능한 것.

불교. 영문도 모른채 태어나서 어쩔 수 없이 살다가 까닭도 모르고 죽는다

사바세계. 인토. 참아야만 하는 세계. 사람이 생각을 하고 선택을 하면 마음대로 안되는 것. 

\subsubsection{3장 부록2. draw a distinction}
독어로는 unterschrieden이라고 하는 것 같음.\\
뭔가 시작을 하려면 definition을 해야 함. social에 대해서 설명을 하는 것. 그거를 또 설명하고 이렇게 하면 시작을 못함. 시작을 한다는 것은 reduction of complexity. 얘기를 못하는게 있는데 그걸 못참으면 시작을 못함. draw a distinction하면 맹점. 안보이는게 생길 수밖에 없음. 모든 걸 보고 알려고 하면 결정과 시작을 못함.

루만. 구분하는 기술. unterschriden technique. there이 아닌게 here. 원래 세상에는 here도 there도 없다. machtspace, unmachtspace. 

순수한 사실성. 사건과 물건. 원래 금이라는 건 없는데, 사람이 구분을 하면서 금이라고 하는 것이 생긴 것. 파악을 하려고 하면 구분을 해야하는데, 구분은 짝짓는 상대성. 상대성의 세계에서는 완전한 해결이 없음. 선을 말하면 악이 따라오고, 악을 말하면 선이 따라옴. 불교의 무념무상은 상대성을 떠나라는 것. 개념을 가지고 하면 상대성의 세계. 인토. 참을 수밖에 없는 세계.

단막증애 통연명백: 무념무상으로 가는 것. 말로 설명하는 것은 구분이 들어오고, 그러면 상대성이 들어오니까. 억지로 말로 표현하려고 하는 것. 구성주의. 상대성이라고 하는 것은 결국 구성(constructivism). 칸트는 dingseich..뭐시기. thing 자체는 모르는 것. 즉 사람은 개념이 아닌 것은 모른다.

나도 쟤가 어떻게 하는가에 따라서 하는데 쟤도 내가 어떻게 하는거에 따라서 하는걸 내가 안다

나도 그런데 쟤도 그래. double이 2개가 있는 것.

이중의 우연성은 사람만이 가능한 것. 

나는 쟤가 하려는 거에 따라서 할랬는데 쟤는 내가할려고 하는거에 따라서 할려고 함. 이것이 패러독스의 역설

공급탄력성: 언어

\subsubsection{4장. 베버}

행위와 행동의 구분. 행위(action)는 의미를 전제로 하는 관찰. 행동(behavior)은 의미 없는 관찰.

stimulus가 response로. 행동주의 심리학에서는 그 안에를 블랙박스로 바라봄. 강화 reinforcement. 

의미라고 하는 것. 주관적으로 의도된 의미. subjectively intention meaning. 

의미가 행위 행위가 성과.

한 명의 학자를 읽었다라고 하는 것은 그 사람을 자신있게 권하면서 그 사람과 대등한 위치에 있어야 한다는 것. 앵무새인 기간을 빨리 벗어나는 것이 좋음. 




\subsubsection{5장. 질서}
자연질서와 사회질서는 다르다. 자연질서는 사람이 아는 한은 바뀌지않음. 변하지 않는 변하지 않음. 사회질서는 신분 사회와 비신분 사회가 다름. 사회질서는 변하는 변하지 않음. 질서에는 불변이 있어야 함. 하지만 사회질서는 사람의 역사에서 바뀐 적이 있음. 예를 들어 신분 사회는 계층 분화인데 비신분사회는 기능분화. reality가 다름. reality가 다르면 사회과학도 불변을 구성해낼 때 달라지는 것. 즉 사회과학은 늙지 않는다. 그 질서를 이루는 것이 주벡티브 게마이티제.

\subsubsection{6장. 추천}

홍원탁. \\
friedrichten tenbruck\\
윤석철.\\
임원택.\\
루만. 베버. 차성환\\
나도 쟤가 어떻게 하는가에 따라서 하는데 쟤도 내가 어떻게 하는거에 따라서 하는걸 내가 안다\\
나도 그런데 쟤도 그래. double이 2개가 있는 것.\\
이중의 우연성은 사람만이 가능한 것. \\
나는 쟤가 하려는 거에 따라서 할랬는데 쟤는 내가할려고 하는거에 따라서 할려고 함. 이것이 패러독스의 역설\\
공급탄력성: 언어\\
"막스베버의 사회과학 방법론"\\
현실탐구과학()\\
칸트를 먹으면 다른걸 다 먹고, 그걸 또 다른걸 도 먹을 수 있음
approaching sociology. 사회학에의 접근\footnote{}\\
단막증애 통연명백.\\
the art of deliberate misunderstanding\\
panta rhei.\\
제행무상 제법무아 열반적정\\
toward a geneeral theory of action\\
플로티노스\\
해체주의\\
스리니바사 라마누잔\\
현대의 사회학. 김경동\\
approaching sociology. 사회학에의 접근. 리들. 박영신.\\
자기 문제를 가지고 있어야 하고, 남의 문제를 자기 문제적으로 해서 자기 안에서, 자발적으로 돌아가야. 그래야 추적이 되고/\\ 그 앵무새의 단계를 넘어가야함\\
힘이 없으면 밟아죽이고 힘이 있으면 띄워서 죽인다\\
학문의 시작은 문제다. \\
자기만의 것이 있어야 한다. 하늘아래 완전히 새로운 것이 없다.\\ koklogische kommunikation\\
guy kirsch neue politische \\
경제학) 개인의 행동 축적 집합 이걸 사회적인 것. 아주\\ 독립적인 것이 아님. 개인으로 한원될 수 있음.\\
사회학) 개인과 뭔가 독립적인것이 있다.\\
집합의식\\
아뢰아식 구정식 부정식 씨앗. 아뢰아씨\\
현대) 뇌를 생각이 아니라 수신장치라고 /// 신호와 소음\\
the world without us 마지막에 남는게 생각. 파동 \\
마크가 남으면 그쪽으로 생각이 몰려옴. 영감을 받는다는 것.\\ <윤홍식의 홍익학당>\\
들뢰즈, 라깡, 해체주의 \\
이정우\\
박홍규 형이상학 프랑스첧가\\
김미정 서울대 사회학과 석사 박사 차이와 윤리\\
프랑스철학. 해체주의 라캉 데리다\\
사회에서의 차이 / 차별 / 차이 \\
불가능성의 정리/ 노벨경제학상 . \\
자기가 자기를 증명도 못하고 반증도 못하는 .\\
의미 / self-referential autoperatic system. 뼈와 살이 되게 함 \\
자기만의 인생경험으로는 안됨. 정수론 인도사람\\
an introduction of theory of numbers\\
같은 물에 두 번 담글 수 없다. 재행무상. 재행무상(상 = invarianz) 모든 행위에는 항상 그러함이 없다.\\ 판타라이(vmffhxlsnt)\\
모든 것은 변한다 / 플로티누스 유동설 유출설.\\ 판타라이(everything flows. 재행무상)\\
toward a general theory of action\\
법\\

[물건이 있는 것 존재]
1법) 이루어진 것. 사건과 사물.. 사건 물건.. \\
2법) 원소. 구성하는 극미세한 모습들. 일체법. 근본적인 요소들이 있을 것이라고 생각.\\
3법) 그런 존재를 법이라고 하면, 그걸 있게 하는 이치들\\
4법) 종교진리. 신을 전제로 . 법칙을 전제로\\
법교법) \\
5법) 법교법. 사건 물건 뒤에 있는 이치이넫, 존재에 관한 걸로 넘어갔는데. 법칙을 전제로 하든가. 그걸로된게 다섯번째\\
6법) 사성재 고진멸도.. 8정도(8종도). \\
7법) 법칙 중에서 부처님이 한 것. 열반에 들어서 체험한 것(=불법)\\
[우리가 아는 것이 마음에 다 있다]\\
8법) 사실 그것도 마음이 떠올리는 일체의 관념. \\
일체유심조\\
그렇게 질서를 바라보는게 해체주의 // 해체주의도 주의는 주의인데.... \\
해체. 해체주의. 컴포넌트. 나누어서 ideal type\\
pragmatism: useful or not\\
모든 것은 변한다라는 건 기능분화사회 / 신분사회를 할 때는 유용하지 않음. True일 수 있지만 useful하지는 않음\\
니클라스 루만 게셀 semantik\\
막스베버 이정우 윤홍식.\\
hobbsian problem of order - 사회적 질서 - 루만\\
커뮤니케이션이 될 수 있게 의미를 모아두는 것. 문화. 그런게 social에 가까움.\\

프리퀀츠. 주파수 \\
개념. 이념. 

쭉 적고나서 옮겨볼게유

대가리 굴리다가 끝나지 않게 하는 아름다운 latex 구조 사용법 ~

\subsection{기본적인 대화의 첫 발단들과 변환의 시초}
\begin{itemize}
    \item 
\end{itemize}

\subsection{대화의 전개 과정에 있어 토대적 이론과 형태}


\begin{itemize}
    \item 
\end{itemize}


\subsection{대화의 소결 혹은 과정의 결과}


\begin{itemize}
    \item 
\end{itemize}


\subsection{이후의 추상과 관념 적인 사변의 연속}


\begin{itemize}
    \item 
\end{itemize}






\newpage
\subsection{정신, 자아 그리고 사회 220807}

\subsubsection{17장 정신이 반응, 환경과 맺는 관계}
(210) 우리가 대상이라고 부르는 것에 대한 태도의 조직화는 우리에게 사물의 의미를 구성하는 것이다... 이 태도들이 서로에게 지니는 관계는 그 속성들에 대한 '실질적 내용'의 관계에 조명을 비춘다. 

(212) 그렇다면 우리는 특별한 정신의 내용이라고 가정되는 곳, 즉 사물의 의미를 행동주의적 진술로 가지고 있다고 할 수 있다. 나는 이 요소들을 태도라고 불러왔다. 

(213) 내가 강조하고 싶은 것은 이 태도가 환경을 결정하는 방식이라는 점이다. 처음에 전보를 보내는 반응들의 조직화된 집합이 있고, 그다음 전달 수단을 선택하고, 뒤이어 출금하기 위해 은행으로 가고, 그러고 나서는 기차에서 읽을 것을 마련한다. 한 세트의 반응들에서 다른 세트로 나아감에 따라 우리는 다음 세트의 반응들에 응답하는 환경을 스스로 선택한다. 하나의 반응을 마무리하는 것은 우리가 다른 것들을 볼 수 있는 위치에 스스로 놓는 것을 뜻한다. 
= 사물의 의미를 행동주의적 진술로 가지고 있는, 그 요소들이 나의 환경을 결정한다. 행동주의적 진술을 강조. 

(213-214) 우리는 공간적으로뿐만 아니라 시간적으로 우리에게서 멀리 떨어져 있는 것들을 본다. 우리가 이것을 할 때 저것을 할 수 있..다. 앞으로 발생하게 될 반응에 의해 우리의 세계에 일치하는 분명한 그림이 우리에게 제공된다. 각주: 환경의 구조는 자연에 대응되는 유기체의 반응을 찾아내는 것이다. 사회적이든 개인적이든 모든 환경은 그것이 대응하는 행위, 즉 외현적 표현을 추구하는 행위의 논리적 구조에 대응하는 일치물이다.

(214) 유기체 안에는 그가 지각하는 외부 대상의 선택적, 상대적 특성을 결정하는 민감성의 구조 또는 형태(gestalt)가 분명하게, 그리고 필연적으로 존재한다. 우리가 의식이라고 부르는 것을 유기체와 환경의 관계 안에 불러들일 필요가 있다. 우리가 환경 중에서 건설적으로 선택하는 색상, 감정적 가치 등과 같은 것은 생리적 민감성 측면에서 볼 때 본질적으로 '의식'이라고 말하는 것이다. 우리는 역사적으로 이 의식을 정신 또는 외 안에서 찾고자 하는 경향이 있었다.

(214-215) 황소가 풀에 음식의 특성을 부여하는 것처럼, 눈과 시각 과정이 정확히 대상에 색상을 부여한다. 그런데 대상에 감각을 투사하는 의미에서가 아니라, 대상의 성질로서 색상의 등장과 존재를 가능하게 해주는 대상과의 관계 안에 스스로 놓는다는 의미에서 그렇다. 지각하는 유기체와의 관계가 있어야만 대상 안에 색상이 담길 수 있는 것이다. 지각하는 유기체의 생리적 또는 감각적 구조가 대상의 경험 내용을 결정한다. 그렇다면 어떤 의미에서 유기체에 의해 환경이 달라지는 것이다. 그리고 유기체와 환경이 서로 결정하기 때문에, 그리고 각자의 존재에 상호 의존적이기 때문에, 적절히 이해하기 위해서는 각자의 생존 과정이 상호관계 속에서 고려되어야 한다. 

(215) 사회적 환경은 사회적 활동의 과정 측면에서 의미를 부여받는다. 그것은 사회적 경험과 행동 과정 안에서 그런 활동에 관여하는 유기체 집단과의 관련 속에 발생하는, 객관적 관계의 조직화다. 외부 세계의 어떤 특성이 소유되는 것은 단지 상호작용하는 개인 유기체들의 사회적 집단 측면에서, 또는 사회적 집단과의 관련 속에서만 가능하다. 다른 특성들도 단지 개인 유기체 자신들과의 관련 아래서만 소유되는 것과 마찬가지다. 행동의 사회적 과정-또는 사회적 유기체-과 사회적 환경의 관계는 개인의 생물학적 활동 과정-또는 개인 유기체-과 물리학적, 생물학적 환경의 관계와 유사하다.

(217) 사람은 자기 자신의 반응들의 관점에서 그 과정을 통제할 수 있다. 선택할 수 있는 인간의 능력은 집을 정신적 사건으로 만든다. 두더지도 음식을 찾아야 하고 적을 만나면 피하지만, 우리는 두더지가 스스로에게 다른 동굴보다 자기 동굴이 특별히 더 이익이 됨을 지시할 수는 없다고 가정한다. 두더지의 집에는 정신적 특성이 없다. 정신성은 유기체가 자기 반응에 대응하는 환경 안에서 다양한 방식으로 반응들을 통제할 수 있도록 그 특성을 지시하는 능력 안에 있다. 행동주의 심리학의 관점에서 볼 때, 바로 거기에 정신성이 존재한다. 환경과 관ㄹ된 행동의 복잡한 요소들이 두더지와 다른 동물들에게도 있지만, 인간은 스스로에게, 그리고 다른 개체에게 이 복잡하고 고도로 조직화된 반응들을 불러일으키는 환경 속의 특성들이 무엇인지를 지시할 수 있고, 이 지시를 통해 반응들을 통제할 수 있다. 

(217-218) 우리 접근에서의 정신성은 단지 유기체가 다른 사람과 자신에게 의미를 지시할 수 있을 때 들어온다. 이것은 정신이 등장하는, 또는 나타나는 지점이다. 우리가 인식해야 할 필요가 있는 것은 유기체가 스스로의 선택성에 의해 선택된 환경과 관련을 맺는다는 사실이다. 심리학자는 인간이 이 관계를 통제하는 데 포함되는 메커니즘에 관심을 둔다. 관계는 지시가 이루어지기 전에 거기에 있지만, 유기체는 스스로의 행위 안에서 그 관계를 통제하는 것이 아니다. 그것은 원래 그것을 통제할 수 있는 수단에 따른 메커니즘을 가지고 있지 않다. 그러나 인간은 이 통제..를 얻을 수 있는 수단으로 언어 커뮤니케이션을 이해해 왔다. 이제 그 메커니즘의 많은 부분이 중추신경계에 있는 것이 아니라 사물이 유기체에게 갖는 관계 안에 있다는 사실이 분명해졌다. 이 의미를 선택하여 그 의미들을 유기체와 다른 개체에게 지시하는 능력은 인간 개체에게 특별한 힘을 부여하는 능력이다. 그 통제는 언어에 의해 가능해진다. 이러한 점에서 언어는 '정신'을 구성하는 의미를 통제하는 메커니즘이다.

(218) 정신을 단순히 개인 인간 유기체의 관점에서 바라보는 것은 불합리하다. 왜냐하면 비록 개인에 초점이 있다 하더라도 본질적으로 그것은 사회적 현상이기 때문이다. 심지어 생물학적 기능조차도 일차적으로는 사회적인 것이다. 개인의 주관적 경험이 정신의 설명으로 가능하게 받아들여질 수 있기 위해서는 뇌의 자연적, 사회생물학적 활동과의 관련 속에서 이해되어야 한다. 그리고 이것은 단지 정신의 사회적 본질을 인식한 경우에만 가능하다. 개인의 경험이 사회적 경험 과정과 분리되면 -사회적 환경과 분리되면- 미약해진다는 것은 분명하다. 그렇다면 우리는 정신이 사회적 과정 속에서, 즉 사회적 상호작용의 경험적 매트릭스 안에서 일어나 발달한다고 간주해야 한다. 즉 우리는 개인이 상호작용하는 사회적 맥락 안에서, 개개인의 경험을 포함하는 사회적 행위의 관점에서 개인의 내적 경험을 얻어야 한다. 

(219) 개인은 정신 과정에 전체로서 지니는 관계를 인식하게 되며, 그와 함께 그 안에 참여하고 있는 다른 개인들에 대한 관계도 인식하게 된다. 그는 그것을 수행하는 자신을 포함한 개인들의 반응과 상호작용에 의해 수정된 과정을 인식하게 된다. 정신이나 지능의 진화적 발생은 경험과 행동의 사회적 과정 전체가 그 안에 함유되어 있는 분리된 개인들 중 한 사람의 경험 속에 들어올 때, 그리고 그 과정에 대한 개인의 적응이 그가 갖게 되는 인식이나 의식에 의해 수정되고 세련화될 때 발생한다. 이와 같이 사회적 과정 전체가 거기 포함되어 있는 개인의 경험 속으로 들어오는 것은 바로 반사라는 수단에 의해서 -개인의 경험을 스스로에게 돌려줌으로써- 가능해진다. 자기를 되돌아보는 바로 그 반사 과정에 의해 개인은 자신에 대한 다른 사람의 태도를 수용하여 의식적으로 그 과정에 스스로 적응할 수 있게 되며, 주어진 사회적 행위 안의 그러한 과정의 결과를 그에 대한 자신의 적응 측면에서 수정할 수 있게 되는 것이다. 그러므로 반사는 사회적 과정 안에서 정신의 발달에 본질적인 조건이다.

반면에 기계는 텍스트와 텍스트 간의 구분을 할 수 있지만, 그 텍스트의 발화적 맥락을 그리고 그 텍스트를 받아들인 자기 스스로의 모습을 적응 측면에서 수정하는 것이 아니다. 그저 기계는 그 텍스트에 대한 올바른 대답을 기존의 텍스트를 통해서 학습할 뿐이다. 즉 인간의 적응, 반사 과정은 자기가 인식하는 것으로서 관계성 안에 놓여져 있지만 기계의 것은 그저 대답에 불과하다. 

\newpage
\subsection{symbolic analysis of relay and switching circuits 220806}

(1-2) Two problems that occur in connection with such networks of switches will be treated here. The first, which will be called
analysis, is to determine the operating characteristics
of a given circuit...The second problem is that of synthesis. Given certain characteristics, it is required to find a circuit incorporating these characteristics. The solution of this type of problem is not unique and it is therefore additionally desirable that the circuit requiring the least number of switch blades and relay contacts be found. Although a solution can usually be
obtained by a "cut and try" method, first satisfying
one requirement and then making additions until all
are satisfied, the circuit so obtained will seldom
be the simplest. This method also has the disadvantages
of being long, and the resulting design often
contains hidden "sneak circuits."\footnote{https://www.sohar.com/sca/whatis-sca.html} \footnote{https://resources.pcb.cadence.com/blog/2020-sneak-circuit-analysis-and-sca-indicators-for-circuit-design}

(2) Any circuit is represented by a set of equations,
the terms of the equations representing the
various relays and switches of the circuit. A calculus
is developed for manipulating these equations
by simple mathematical processes, most of which are
similar to ordinary algebraic algorisms. This calculus
is shown to be exactly analogous to the Calculus
of Propositions used in the symbolic study of
logic. \\

\newpage

\subsection{The reckoning accountability}

(14) 전제군주들은 왜 투명한 회계를 두려워했나?라는 질문에서 전제군주들은 왜 회계를 용납하였나?라는 질문을 제기할 수 있다. 이 질문은 본문 14페이지의 ''르네상스와 그 뒤에 등장한 과학혁명의 절정기인 1480년과 1700년 사이에 왕들은 회계에 관심을 가졌다"가 어떻게 가능한가라는 질문의 대답과 같다. \\

(19) 이 사상가들은 회계를 경제적 성공의 필수 요소이자 경제사를 이해하는 열쇠라고 보았다. 그러나 그들이 보지 못한 것은 회계가 단지 경제적 문제뿐 아니라 정치에도 영향을 미친다는 사실이었다. 정치적 안전성은 책임성의 문화를 토대로 하며, 그런 책임성은 바로 복식부기 회계 제도에 의존한다. 의 내용에서 복식부기 회계 제도가 책임성의 문화 그리고 그게 정치적 안전성으로 이어진다는 논리에 대한 충분한 설명을 찾거나 듣고싶음. \\ 

(27) 최상의 투명성은 언제나 옳은 가치일까? 엄격한 감사는 선이고 어느 정도의 기만행위는 무조건 악인가? 책의 각주\footnote{Boecke, The Public Economy of Athens, p. 194} 

(40) 최초의 복식장부는 하나의 문단을 이루는 문장 형식이었으며 차변 문단과 대변 문단이 상응했다. 그러던 것이 나중에는 문단과 서술 대신 순수하게 숫자만 이용해 나란히 붙은 두 개의 열에 기록하는 이원적(bilateral) 형식으로 바뀌었다. 최초의 복식 기장의 예는 1340년 제노바 중앙원장에 기록된, 제노바 상인 야코부스 데 보니카(Jacobus De Bonicha)의 수익성 좋은 후추 교역 거래에 관한 것이다. 네덜란드가 아니라 이탈리아 남부의 제노바.

(42) 그러나 이 시스템은 혁신적이고 효과적이었음에도 불구하고 르네상스의 이탈리아를 넘어 뻗어나가지 못했다. 북쪽의 전제군주들은 옛 상업 공화정의 재산관리 방식을 채택하는데 매우 더뎠다. 복식 기장이 중앙 정부의 원장과 종합적 국가 재무 감사를 위한 관리 도구로 다시 이용되기까지 무려 600여 년이 걸렸다. 유럽 정부들이 효과적으로 회계를 이용하기 전까지, 르네상스 사상가들은 재무 질성의 필요와 돈 계산을 부도덕하게 여기는 기독교적 사고방식 사이에서 균형을 잡아야 했다. 왜 북쪽의 전제군주들은 매우 더뎠나? 책임을 지는 것이 안정적인 질서의 기반이라면, 왜 그러지 않았나? 그러헥 채택하지 않은 것이 이탈리아의 수많은 도시국가 간 분쟁을 낳은 것은 아닐 것 같은데?

(139) 부패는 개인에 기인하는가, 시스템적인 문제인가? 4장 스페인을 읽고나서. 

(188) 돈을 잘 관리하는 것과 회계는 어떤 차이가 있는지? 돈의 입출납을 잘 보고 그거에 맞게 씀씀이를 하는 것이 안녕의 전부인가? 루이 14세처럼 회계장부에 신경쓰는 것이 아니라 내가 하고싶은대로 하는 것. 장부에 신경쓰지 않는 것이 오히려 전제군주라는 점을 고려했을 때는 더욱 당연한 것은 아닐까? 왜 국정 전체를 국왕이 신경써야하는게 당연한 것이 될까?


(260) 저자는 회계에서 책임성이라고 하는 단어의 비중을 상당히 높게 가져가는 생각이 들음. 자기가 스스로 책임성을 다하는 것은 도덕적으로나 실천적으로나 상당히 어려운 일임. 책임성은 규범적인 것일까?

(277) 베버는 즉각적인 자기만족을 부정하고 돈을 버는 것은 단순히 좋은 자본주의적 도구일 뿐 아니라 신성한 칼뱅주의 윤리라고 결론 지으며 성경 구절을 인용했다. "네가 자기의 일에 능숙한 사람을 보았느냐? 이러한 사람은 왕 앞에 설 것이요, 천한 자 앞에 서지 아니하리라" 10장. 주 7.\footnote{프로테스탄티즘 윤리pp50-67} 이 내용은 사실관계를 다시 따져볼 필요가 있음. 베버에 있어서 결론을 내린다라는 것은 상당히 위험한 것이며 오히려 돈을 추구하는 것이 도구합리적 행위로 부정적인 것.

(323) 이 책에서 12장은 회계의 합리성을 의심하는 찰스 디킨스의 말로 시작함으로써 다른 장과는 구별되는 특성을 보임

(323-324) 1828년 소셜 금치산 선고에서 프랑스 작가 오노레드 발자크는 회계가 '인간 마음의 비참함'을 가늠하기에 최적화된 도구임을 보여주었다. 발자크는 파리의 치안판사 포피노가 어떻게 재무사기를 조사했는지, 그리고 어떻게 바스티유 광장 바로 윗동네 파리 제12구 주민들의 삶을 처리하는 회계 시스템을 개발했는지를 묘사했다. "그 동네의 모든 고통에 번호를 매기고, 마치 다양한 채무자를 기록하는 상인처럼 각각의 불행을 계정으로 만들어 장부에 기록했다." 그의 시스템은 재정도, 도덕적 권리도, 잘못도, 심지어 행복도 측정하지 않았다. 말하자면 콜베르의 경우처럼, 그것은 경찰 활동의 도구였다.

(331-332) 다윈의 일화는 회계의 내용과 전혀 맞지 않는 것으로 판단됨.

(361) 13장의 현대의 회계부정사건은 과거의 역사에 비추어 필연적인 비극을 맞이할 것임을 제시 및 서술하고 있음과 동시에 미래에 대한 메시지를 담고 있다고 해석됨. 책무성의 부재는 개인의 문제인가? 

(369) 책임성을 이루기 위해 싸워온 역사

실패가 내재되어 있는 세계 금융 시스템(370 부제)

(372) 여기서 배울 역사적 교훈이 있다면, 회계를 전반적 문화의 일부로 이용할 수 있었던 사회는 번성했다는 것이다. 이 책에서 논의한 곳들만 예로 들면, 제노바와 피렌체 같은 이탈리아 도시공화국과 황금기의 네덜란드, 18세기와 19세기의 영국과 미국은 모두 회계를 교육 과정과 종교적, 도덕적 사상과 예술, 철학, 정치 이론에 통합시켰다.

(373-374) 어쩌면 지나친 물질주의 시대에 휘청거리는 우리 사회를 구원할 방법은, 조사이어 웨지우드의 개인적이고 규율적인 회계나 애덤 스미스 같은 경제사상가들의 역사적이고 도덕적인 접근법과 근대 계산기의 분석 속에 있을지 모른다. 혹은 회계를 신앙과 윤리, 시민 정치, 예술 속에 단단히 잡아둠으로써 장부 기록과 재무 관리의 중요성을 강렬하게 보여주는 얀 프로보스트의 <죽음과 수전노> 같은 그림에 담긴 오랜 교훈 속에 그 방법이 있을지도 모른다.

(374) 우리는 한때 재무에 대해 사유하고 그것을 실행하는 사람들에게, 회계 수치를 사회와 문화의 필수적인 부분으로 여기고 회계장부의 세속적 수치들에서 종교적, 문학적 의미를 읽어내라고 요구했다. 미래의 심판을 대면하기 위해 우리는 바로 이런 문화적 포부를 되찾아야 할 것이다.

아까 그 질문 정도가 끝.

처음의 그 것, 부패(책임성)은 개인의 문제인지? 투명성은 언제나 최선의 가치인가?


\end{document}