\documentclass[11pt, a4paper]{article}
\usepackage[utf8]{inputenc}
\usepackage{fancyhdr}
\usepackage{graphicx}
\usepackage[parfill]{parskip}
\PassOptionsToPackage{hyphens}{url}\usepackage{hyperref}
\usepackage{geometry}
\usepackage[
backend=biber,
style=ieee,
]{biblatex}
\usepackage{kotex}


\pagestyle{fancy}
\fancyhf{}
\setlength{\headheight}{14pt}
\rhead{논문}
\lhead{박석훈}
\cfoot{\thepage}

\begin{document}

\title{생각하고 보고 생각}
\author{Sukhoon Park}
\maketitle

\begin{abstract}

2022. 08. 06. \\
발단: 이거 해야해. 이거 해야해? 이거 굳이 해야해? \\
전개: 이건 뭘 의미해? 이런거 해야해? \\
이 개념은 무엇을 의미해? \\
반대로 이 개념은 뭘 의미하고 있지 않는거야? \\
이 개념은 무엇과 유사해? 그리고 유사한 무엇과는 어떤 부분에서 달라? \\
이 개념은 반드시 존재해야해? \\
이 개념이 잘못 이해하거나 되면 어떤 부작용이 생겨? \\
위기: 이거 할 수 있는거 맞아? \\
절정: 몰라 하래서 한거자나 \\
종말: 어어어어 \\

2022. 08. 07.

내적 행복함수를 찾기. 사상을 대가로 돈을 받았을 때 연명하게 될 학자들의 모습. \\


\end{abstract}


\newpage

\subsection{symbolic analysis of relay and switching circuits 220806}

(1-2) Two problems that occur in connection with such networks of switches will be treated here. The first, which will be called
analysis, is to determine the operating characteristics
of a given circuit...The second problem is that of synthesis. Given certain characteristics, it is required to find a circuit incorporating these characteristics. The solution of this type of problem is not unique and it is therefore additionally desirable that the circuit requiring the least number of switch blades and relay contacts be found. Although a solution can usually be
obtained by a "cut and try" method, first satisfying
one requirement and then making additions until all
are satisfied, the circuit so obtained will seldom
be the simplest. This method also has the disadvantages
of being long, and the resulting design often
contains hidden "sneak circuits."\footnote{https://www.sohar.com/sca/whatis-sca.html} \footnote{https://resources.pcb.cadence.com/blog/2020-sneak-circuit-analysis-and-sca-indicators-for-circuit-design}

(2) Any circuit is represented by a set of equations,
the terms of the equations representing the
various relays and switches of the circuit. A calculus
is developed for manipulating these equations
by simple mathematical processes, most of which are
similar to ordinary algebraic algorisms. This calculus
is shown to be exactly analogous to the Calculus
of Propositions used in the symbolic study of
logic. \\

\newpage

\subsection{The reckoning accountability}

(14) 전제군주들은 왜 투명한 회계를 두려워했나?라는 질문에서 전제군주들은 왜 회계를 용납하였나?라는 질문을 제기할 수 있다. 이 질문은 본문 14페이지의 ''르네상스와 그 뒤에 등장한 과학혁명의 절정기인 1480년과 1700년 사이에 왕들은 회계에 관심을 가졌다"가 어떻게 가능한가라는 질문의 대답과 같다. \\

(19) 이 사상가들은 회계를 경제적 성공의 필수 요소이자 경제사를 이해하는 열쇠라고 보았다. 그러나 그들이 보지 못한 것은 회계가 단지 경제적 문제뿐 아니라 정치에도 영향을 미친다는 사실이었다. 정치적 안전성은 책임성의 문화를 토대로 하며, 그런 책임성은 바로 복식부기 회계 제도에 의존한다. 의 내용에서 복식부기 회계 제도가 책임성의 문화 그리고 그게 정치적 안전성으로 이어진다는 논리에 대한 충분한 설명을 찾거나 듣고싶음. \\ 

(27) 최상의 투명성은 언제나 옳은 가치일까? 엄격한 감사는 선이고 어느 정도의 기만행위는 무조건 악인가? 책의 각주\footnote{Boecke, The Public Economy of Athens, p. 194} 

(40) 최초의 복식장부는 하나의 문단을 이루는 문장 형식이었으며 차변 문단과 대변 문단이 상응했다. 그러던 것이 나중에는 문단과 서술 대신 순수하게 숫자만 이용해 나란히 붙은 두 개의 열에 기록하는 이원적(bilateral) 형식으로 바뀌었다. 최초의 복식 기장의 예는 1340년 제노바 중앙원장에 기록된, 제노바 상인 야코부스 데 보니카(Jacobus De Bonicha)의 수익성 좋은 후추 교역 거래에 관한 것이다. 네덜란드가 아니라 이탈리아 남부의 제노바.

(42) 그러나 이 시스템은 혁신적이고 효과적이었음에도 불구하고 르네상스의 이탈리아를 넘어 뻗어나가지 못했다. 북쪽의 전제군주들은 옛 상업 공화정의 재산관리 방식을 채택하는데 매우 더뎠다. 복식 기장이 중앙 정부의 원장과 종합적 국가 재무 감사를 위한 관리 도구로 다시 이용되기까지 무려 600여 년이 걸렸다. 유럽 정부들이 효과적으로 회계를 이용하기 전까지, 르네상스 사상가들은 재무 질성의 필요와 돈 계산을 부도덕하게 여기는 기독교적 사고방식 사이에서 균형을 잡아야 했다. 왜 북쪽의 전제군주들은 매우 더뎠나? 책임을 지는 것이 안정적인 질서의 기반이라면, 왜 그러지 않았나? 그러헥 채택하지 않은 것이 이탈리아의 수많은 도시국가 간 분쟁을 낳은 것은 아닐 것 같은데?

(139) 부패는 개인에 기인하는가, 시스템적인 문제인가? 4장 스페인을 읽고나서. 

(188) 돈을 잘 관리하는 것과 회계는 어떤 차이가 있는지? 돈의 입출납을 잘 보고 그거에 맞게 씀씀이를 하는 것이 안녕의 전부인가? 루이 14세처럼 회계장부에 신경쓰는 것이 아니라 내가 하고싶은대로 하는 것. 장부에 신경쓰지 않는 것이 오히려 전제군주라는 점을 고려했을 때는 더욱 당연한 것은 아닐까? 왜 국정 전체를 국왕이 신경써야하는게 당연한 것이 될까?









\end{document}